\documentclass[a4paper]{paper}
\usepackage{amsmath,amsfonts,mathrsfs}
%\usepackage{tikz-cd}
\usepackage[matrix,arrow]{xy}
\usepackage{xspace}
\usepackage{graphicx}
\usepackage{paralist}
\usepackage{hyperref} 
\hypersetup{%
    bookmarks=true,         % show bookmarks bar?
    pdfmenubar=true,       % show Acrobat's menu?
    pdffitwindow=false,     % window fit to page when opened
    pdfstartview={FitH},    % fits the width of the page to the window
    colorlinks=true,       % false: boxed links; true: colored links
    linkcolor=red,          % color of internal links
    citecolor=red,        % color of links to bibliography
    filecolor=magenta,      % color of file links
    urlcolor=cyan,           % color of external links
    pdfborder = {0,0,0}
}
\usepackage{acro}
\usepackage{extraenv}
\usepackage{cleveref}
\usepackage{todonotes}


% Specific packages
\newcommand{\Discr}{\mathfrak{D}}
\newcommand{\Spc}[1]{\mathscr{#1}}
\newcommand{\Smooth}{\mathscr{C}}
\newcommand{\Lebegue}{\mathscr{L}}
\newcommand{\Soboloev}{\mathscr{W}}
\newcommand{\Field}{\mathbb{F}}
\newcommand{\Real}{\mathbb{R}}
\newcommand{\Complex}{\mathbb{C}}
\newcommand{\Natural}{\mathbb{N}}
\newcommand{\imagedomain}{\Omega}
\newcommand{\imagerange}{V}
\newcommand{\Op}[1]{\mathcal{#1}}
\newcommand{\DiscOp}[1]{\mathsf{#1}}
\newcommand*{\EXT}[2]{\ensuremath{E_{#1}^{#2}}}
\newcommand*{\REST}[2]{\ensuremath{R_{#1}^{#2}}}
\newcommand*{\PROJ}[2]{\ensuremath{P_{#1}^{#2}}}
\newcommand*{\COPROJ}[2]{\ensuremath{Q_{#1}^{#2}}}
\newcommand*{\RnX}{\ensuremath{\REST{n}{\Spc{X}}}}
\newcommand*{\RmX}{\ensuremath{\REST{m}{\Spc{X}}}}
\newcommand*{\RmY}{\ensuremath{\REST{m}{\Spc{Y}}}}
\newcommand*{\EnX}{\ensuremath{\EXT{n}{\Spc{X}}}}
\newcommand*{\EmX}{\ensuremath{\EXT{m}{\Spc{X}}}}
\newcommand*{\EmY}{\ensuremath{\EXT{m}{\Spc{Y}}}}
\newcommand*{\PnX}{\ensuremath{\PROJ{n}{\Spc{X}}}}
\newcommand*{\PmY}{\ensuremath{\PROJ{m}{\Spc{Y}}}}
\newcommand*{\QnX}{\ensuremath{\COPROJ{n}{\Spc{X}}}}
\newcommand*{\QmY}{\ensuremath{\COPROJ{m}{\Spc{Y}}}}
\DeclareMathOperator{\rest}{\pi}
\DeclareMathOperator{\extend}{\epsilon}
\DeclareMathOperator{\range}{Range}
\DeclareMathOperator{\kernel}{Ker}
\DeclareMathOperator{\nullspace}{N}
\DeclareMathOperator{\Hom}{Hom}
\DeclareMathOperator{\Map}{\mathcal{M}}
\DeclareMathOperator{\Id}{Id}
\DeclareMathOperator{\DIM}{dim}
\newcommand{\ip}[2]{\left \langle #1,#2 \right\rangle}
\DeclareMathOperator{\linspan}{linspan}
\newcommand*{\FUNCRESTR}[2]{{\ensuremath{\left. \vphantom{g_{g_g}} {#1} \right|_{#2}}}}
\newcommand*{\STACK}[2]{\genfrac{}{}{0pt}{1}{#1}{#2}}
\newcommand*{\INNER}[2]{\ensuremath{\langle #1,\,#2\rangle}}
\newcommand*{\INNERbig}[2]{\ensuremath{\big\langle #1,\,#2\big\rangle}}
\newcommand*{\INNERBig}[2]{\ensuremath{\Big\langle #1,\,#2\Big\rangle}}
\newcommand*{\INNERLR}[2]{\ensuremath{\left\langle #1,\,#2\right\rangle}}

\newcommand{\ie}{\textsl{i.e.}\xspace}
\newcommand{\eg}{\textsl{e.g.}\xspace}
\newcommand{\etc}{\textsl{e.t.c.}\xspace}
\newcommand{\wrt}{{w.r.t.}\@\xspace}


\newcommand{\vzero}{\boldsymbol{0}}
\newcommand{\ve}{\boldsymbol{e}}
\newcommand{\vf}{\boldsymbol{f}}
\newcommand{\vg}{\boldsymbol{g}}
\newcommand{\valpha}{\boldsymbol{\alpha}}
\newcommand{\vbeta}{\boldsymbol{\beta}}
\newcommand{\vnu}{\boldsymbol{\nu}}
\newcommand{\vA}{\boldsymbol{A}}

\newcommand{\cF}{\mathcal{F}}

\newcommand*{\NORMLR}[1]{\ensuremath{\left\lVert #1 \right\rVert}}
\newcommand*{\NORM}[1]{\ensuremath{\lVert #1 \rVert}}
\newcommand*{\NOTE}[2][\null]{%
  \marginpar{\renewcommand{\baselinestretch}{1}\vspace{-1em}\hrule\vspace{3pt}%
  \scriptsize\raggedright\textsf{#2\ifx#1\null\else\\\hfill---
  {\em #1}\fi}\vspace{1.5em}}%
}

\newcommand{\ext}{\text{ext}}
\newcommand{\wt}[1]{\widetilde{#1}}



\title{The discretization problem and associated software type system}
\author{Ozan \"Oktem \and Holger Kohr}

\begin{document}
\maketitle

\section{Introduction}
\label{sec:introduction}
%
In many applications one works with mappings between infinite-dimensional vector spaces.
To numerically deal with such mappings, one is forced to find a finite-dimensional 
counterpart. The process of mapping a vector space to a finite-dimensional one is called
\emph{discretization}. A discretization of a specific space is intimately 
tied to the numerical methodology that one eventually wishes to use. However, many concepts
and results about discretizations can be stated in a general mathematical framework.  

Most complex numerical software packages work with several different discretizations, even
of the same vector space. The general mathematical theory that formalizes the process of
discretization yields consistency checks and recipes for how to switch between two
discretizations. Still, most numerical software packages do not utilize such consistency
checks and, even though changes between two discretizations are explicitly coded, this
code is often scattered throughout or even repeated in different parts of the software package.  

To minimize software bugs, it is therefore desirable to be able to abstract and isolate those parts 
of the code that deal with the discretization. To achieve this, the software must be able to represent 
the abstract process of discretization, and an object oriented framework appears to be most
suitable. We will in this note describe those parts of the abstract mathematical theory that
need to be captured by a numerical software package in able to achieve the above stated goals.
We start by giving the basic mathematical foundations.

In many cases, such as inverse problems arising in applications, one has to deal with 
the following problem: Let $\Spc{X}$ and $\Spc{Y}$ be infinite-dimensional vector spaces and 
$\Op{T} \colon \Spc{X} \to \Spc{Y}$. Consider now the following problem:
%
\begin{equation}
 \label{eq:operator_equation}
 \text{Given $g\in \Spc{Y}$, find $f\in \Spc{X}$ such that}\quad \Op{T}(f)=g. 
\end{equation}
%
The operator $\Op{T}$ can either be given explicitly by a closed form expression, or it can be defined 
implicitly, \eg, as a solution operator to a differential equation. 

In order to numerically work with \eqref{eq:operator_equation} in a software package, one must replace the 
infinite-dimensional vector spaces $\Spc{X}$ and $\Spc{Y}$ with some finite-dimensional 
counterparts $\Field^{n}$ and $\Field^{m}$ with $\Field = \Real \text{ or } \Complex$, and the map 
$\Op{T} \colon \Spc{X} \to \Spc{Y}$ with an appropriate discretized operator 
$\DiscOp{T}_{m,n}$. This process yields the following system of (linear or nonlinear) equations
with finitely many unknowns:
%
\begin{equation}
 \label{eq:discrete_operator_equation}
 \text{Given $\vbeta \in \Field^{m}$, find $\valpha \in \Field^{n}$ such that }\quad \DiscOp{T}_{n,m} (\valpha)=\vbeta. 
\end{equation}
%
The above process of replacing \eqref{eq:operator_equation} with \eqref{eq:discrete_operator_equation} is referred to 
as \emph{discretization} of \eqref{eq:operator_equation}. A general theory of discretizations is outlined in 
\cite[Chapter~34]{ZeIIB85}, see also \cite{Pe93}, but the results are of little use for ill-posed inverse 
problems since the discretized problem \eqref{eq:discrete_operator_equation} does not fulfill the requirements.
\emph{In this note we only consider a discretization of \eqref{eq:operator_equation} 
as a way of reducing an equation in infinite-dimensional setting to an equation 
\eqref{eq:discrete_operator_equation} in a finite-dimensional setting.}
Thus, we now formally introduce the notion of discretization in different contexts along with elementary properties.


\section{Discretization and corresponding operators}
\label{sec:discretization_compatibility}

\subsection{Discretization of vector spaces}
\label{subsec:space_discretization}

We start out by defining the discretization of a (infinite-dimensional) vector space with the help of 
reduction and extension operators.

\begin{definition}
 \label{def:space_discretization}
 Let $\Spc{X}$ be a vector space over a field $\Field = \Real \text{ or } \Complex$, and $n \in \Natural$.
 An \emph{$n$-dimensional discretization $\Discr_{n}(\Spc{X})$ of $\Spc{X}$} is the tuple
 %
 \begin{equation*}
  \Discr_{n}(\Spc{X}) := \bigl( \Spc{X}, \Field^{n}, \RnX, \EnX \bigr) 
 \end{equation*}
 %
 where $\RnX \colon \Spc{X} \to \Field^{n}$ and $\EnX \colon \Field^{n} \to \Spc{X}$. The mappings $\RnX$ and $\EnX$ 
 are called the \emph{restriction} and \emph{extension} operators associated with the discretization. 
 $\Discr_n(\Spc{X})$ is a \emph{linear discretization} whenever both these mappings are linear.\\
 %
 There are two additional important mappings derived from $\RnX$ and $\EnX$,
 %
 \begin{align*}
  & \PnX \colon \Spc{X} \to \Spc{X} \quad \text{with} \quad \PnX := \RnX \circ \EnX, \\
  & \QnX \colon \Field^n \to \Field^n \quad \text{with} \quad \QnX := \EnX \circ \RnX,
 \end{align*}
 %
 which we refer to as \emph{projection} operators associated with the discretization (although they are not 
necessarily  projections in the strict sense).
\end{definition}

\begin{examp}
 We begin with considering the case where $\Spc{X}$ is an $n$-dimensional vector space over $\Field$. Then $\Spc{X}$ is 
 isomorphic to $\Field^{n}$, so we may choose $\EnX:=\RnX:=\Id$. The resulting discretization $( \Field^n, \Field^n, 
\Id, \Id )$ is called 
 the \emph{identity discretization}. 
\end{examp}

\begin{examp}
 Another common discretization for an $n$-dimensional vector space $\Spc{X}$ is to consider a basis 
 $B:=\{\psi_1,\ldots, \psi_{n}\}$ of $\Spc{X}$. Any element $f \in \Spc{X}$ can then be written as 
 $f = \sum_{i=1}^n \alpha_i \psi_i$ for unique $\valpha(f) = (\alpha_1, \ldots, \alpha_{n})\in\Field^n$, so we define
 %
 \begin{equation*}
  \RnX(f) := \valpha(f) \quad \text{and} \quad \EnX(\beta) := \sum_{i=1}^n \beta_i \psi_i. 
 \end{equation*}
 %
 The resulting discretization $( \Spc{X}, \Field^{n}, \RnX, \EnX )$ is called the \emph{standard discretization \wrt 
 the basis $B$}. The projection operators are given by
 %
 \begin{equation*}
  \PnX(f) := \sum_{i=1}^n \valpha_i(f)\, \psi_i \quad \text{and} \quad \QnX(\beta) := \vbeta
 \end{equation*}

\end{examp}

\begin{examp}
 A common way to discretize a (possibly infinite-dimensional) vector space $\Spc{X}$ is as follows. Let 
 $\psi_1, \ldots, \psi_n \in \Spc{X}$ be linearly independent space elements and 
 $\lambda_1, \ldots, \lambda_n \in \Spc{X}^*$ be linearly independent functionals on $\Spc{X}$. Then, restriction and 
 extension operators can be defined as
 %
 \begin{equation*}
  \RnX(f) := \big( \lambda_1(f), \ldots, \lambda_n(f) \big) 
  \quad \text{and} \quad
  \EnX(\valpha) := \sum_{i=1}^n \alpha_i\, \psi_i.
 \end{equation*}
 %
 The corresponding projection operators are
 %
 \begin{equation*}
  \PnX(f) := \sum_{i=1}^n \lambda_i(f)\, \psi_i
  \quad \text{and} \quad
  \EnX(\valpha) := \bigg( \sum_{j=1}^n \alpha_j\, \lambda_i(\psi_j) \bigg)_{i=1}^n.
 \end{equation*}

\end{examp}


\subsection{Discretization of product spaces}
\label{subsec:product_space_discretization}

In mathematics it is rather common to construct a new vector space from a collection of vector spaces. If the vector 
spaces in this 
collection all have discretizations, then one can use these discretizations to define a natural discretization on the 
aforementioned new 
vector space.

\begin{definition}
 \label{def:product_space_discretization}
 Let $\Spc{X}_i$ for $i=1,\ldots,k$ be vector spaces over $\Field$ with $n_i$-dimensional discretizations 
 $\Discr_{n_i}(\Spc{X}_i)$ given as
 %
 \begin{equation*}
  \Discr_{n_i}(\Spc{X}_i) := \bigl( \Spc{X}_i, \Field^{n_i}, \REST{n_i}{\Spc{X}_i}, \EXT{n_i}{\Spc{X}_i} \bigr). 
 \end{equation*}
 %
 Then the \emph{product discretization of $\Spc{X} := \Spc{X}_1 \times \ldots \times \Spc{X}_k$} is defined as 
 %
 \begin{equation*}
  \Discr_{N}(\Spc{X}):=\{\Spc{X}, \Field^N, R_{N}^{\Spc{X}}, E_{N}^{\Spc{X}} \}
 \end{equation*}
 %
 with $N := n_1 + \ldots + n_k$, where
 %
 \begin{align*}
  \REST{N}{\Spc{X}}(f_1,\ldots,f_k) &:= \bigl( \REST{n_1}{\Spc{X}_1}(f_1),\ldots,\REST{n_k}{\Spc{X}_k}(f_k) \bigr)
   \quad\text{for $f_i\in \Spc{X}_i$, $i=1,\ldots,k$} \\
  \EXT{N}{\Spc{X}}(f_1,\ldots,f_k) &:= \bigl( \EXT{n_1}{\Spc{X}_1}(\valpha_1),\ldots,\EXT{n_k}{\Spc{X}_k}(\valpha_k) 
\bigr)
   \quad\text{for $\valpha_i\in \Field^{n_i}$, $i=1,\ldots,k$.}
 \end{align*}
\end{definition}

%
Common special cases of product discretizations are $\Spc{X}:=\Field \times \Spc{X}_2$, which corresponds to the case 
$k=2$ and $\Spc{X}_1=\Field$ with the identity discretization.


\subsection{Natural discretization of operators}

When given an operator $\Op{T}$ between vector spaces $\Spc{X}$ and $\Spc{Y}$, each with a given discretization, 
there is a natural discretized counterpart $\DiscOp{T} \colon \Field^n \to \Field^m$ defined as follows:

\begin{definition}
 \label{def:operator_discretization}
 Let $\Op{T}\colon \Spc{X} \to \Spc{Y}$ be an operator and $\Discr_n(\Spc{X})$ and $\Discr_m(\Spc{Y})$ be 
 discretizations of the vector spaces as in Definition~\ref{def:space_discretization}. The \emph{natural operator 
 discretization of $\Op{T}$} is defined as $\DiscOp{T} \colon \Field^n \to \Field^m$ given as 
 $\DiscOp{T} := \RmY \circ \Op{T} \circ \EnX$.
\end{definition}

%Let us now turn to discretization of equations in Banach spaces.
%\begin{definition} 
%  Let $\Spc{X}$ be an infinite-dimensional Banach space with norm $\Vert \cdot \Vert_{\Spc{X}}$. 
%  Assume that for each $n=1,2,\ldots$ we are given an $n$-dimensional  discretization
%  \[ \Discr_{n} := \{ \Spc{X}, \Field^{n}, \RnX,\EnX \}. \]
%  Then we say that the  sequence $\{ \Discr_{n} \}_{n}$ of $n$-dimensional discretizations 
%  of $\Spc{X}$ is \emph{compatible} whenever
%  \[ \lim_{n\to \infty} 
%     \bigl\Vert ( \EnX\circ \RnX)(f)-f \bigr\Vert_{\Spc{X}}\to 0
%    \quad\text{for all $f\in \Spc{X}$.}\]
%\end{definition}
%Thus, if one has a sequence $\{ \Discr_{n} \}_{n}$ of $n$-dimensional discretizations 
%of $\Spc{X}$, then for large $n$ one can make the approximation that 
%$\EnX\circ  \RnX$ is the identity operator, \ie,
%\begin{equation}\label{eq:fApprox}
%  f \approx \EnX(f_{n}) \quad\text{where}\quad
%  f_{n}:=\RnX(f).
%\end{equation}
%\begin{definition}
%  Let $\Spc{X}$ and $\Spc{Y}$ be Hilbert spaces over the field $\Field$ and $\Op{T} \colon \Spc{X} \to \Spc{Y}$. 
%  Also, let 
%  \[  \Discr_{n}(\Spc{X}):=\bigl\{ \Spc{X},\Field^{n},\RnX,\EnX \bigr\} 
%     \quad\text{and}\quad
%     \Discr_{m}(\Spc{Y}):=\bigl\{ \Spc{Y},\Field^{m},R_{m},E_{m} \bigr\}\]
%  denote $n$- and $m$-dimensional discretizations of $\Spc{X}$ and $\Spc{Y}$, 
%  respectively. Then a map $\DiscOp{T} \colon \Field^{n} \to \Field^{m}$ is the
%  \emph{discretization of $\Op{T}$ \wrt $\Discr_{n}(\Spc{X})$ and $\Discr_{m}(\Spc{Y})$},  
%  whenever 
%  \[ \DiscOp{T} = \RmY \circ \Op{T} \circ \EnX \quad\text{or}\quad
%    \Op{T} = \EmY \circ \DiscOp{T} \circ \RnX. \]
%  The problem of  solving the equation
%   \[  \Op{T}(f)=g \quad\text{for $f\in \Spc{X}$ when $g\in \Spc{Y}$ is given,} \]
%  is \emph{discretized} as solving 
%  \[   \DiscOp{T}(a)=b \quad\text{for $a\in \Field^{n}$ when $b\in \Field^{m}$ is given.} \]
%\end{definition}  
%\begin{remark}
%Note that we do \emph{not require that both  
%\[ \DiscOp{T} = \RmY \circ \Op{T} \circ \EnX \quad\text{and}\quad
%    \Op{T} = \EmY \circ \DiscOp{T} \circ \RnX \]
%holds}. This would imply that  $\RmY$ and $\RnX$ are the inverses of $\EmY$ and 
%$\EnX$, respectively, which is way to strong. Moreover, note that the requirement that 
%$\DiscOp{T}$ is a discretization of $\Op{T}$ is stronger than  the requirement that 
%$\Op{T}$ and $\DiscOp{T}$ correspond to each other.
%\end{remark}
%
%\begin{examp}
%Let us consider a simple example which is the 
%Galerkin method in Hilbert space $\Spc{X}$. Let $\{ \phi_{i} \}_{i}\subset \Spc{X}$ be a 
%complete orthonormal system in the separable Hilbert space $\Spc{X}$ and 
%$\Op{T} \colon \Spc{X} \to \Spc{X}$. 
%Let $\RnX \colon \Spc{X} \to \Field^{n}$ be the orthogonal projection operator 
%from $\Spc{X}$ onto $\Field^{n}$, \ie,
%\[ \RnX(f)= \bigl( \ip{\phi_{1}}{f}_{\Spc{X}}, \ldots, \ip{\phi_{n}}{f}_{\Spc{X}} \bigr)
%   \quad\text{for all $f\in \Spc{X}$.} \]
%Moreover, let $\EnX\colon \Field^{n} \to \Spc{X}$ be the corresponding embedding operator, so
%\[ \EnX(\alpha)= \sum_{i?1}^n \alpha_i \phi_{i}
%   \quad\text{for all $\alpha\in \Field^{n}$.} \]
%Then, $\RnX(f)\to f$ as $n\to \infty$, so the discretizations of $\Spc{X}$ are 
%compatible. Moreover, the discretization $\DiscOp{T}_{n,n}$ of $\Op{T}$ where 
%\[ \DiscOp{T}_{n,n}:=\RnX\circ \Op{T} \circ \EnX \]
%is consistent with the discretization of $\Spc{X}$ and the equation
%$\Op{T}(f)=g$ can be replaced with 
%\[ \ip{\DiscOp{T}(f_{n})}{\phi_{j}}_{\Spc{X}}=\ip{g}{\phi_{j}}_{\Spc{X}}
%   \quad\text{$f_{n}\in \Field^{n}$ and $j=1,\ldots,n$.} \]
%\end{examp}


\subsubsection{Natural discretization of spaces of linear operators}
The above notion of a natural operator discretization, we immediately get a natural discretization of the space of 
linear operators from $\Spc{X}$ to $\Spc{Y}$.

\begin{definition}
 \label{def:operator_space_discretization}
 Let $\Spc{X}$ and $\Spc{Y}$ be vector spaces with discretizations $\Discr_n(\Spc{X})$ and $\Discr_m(\Spc{Y})$ 
 as in Definition~\ref{def:space_discretization}. Let further $\Spc{M} := L(\Spc{X}, \Spc{Y})$ be the vector 
 space of linear operators from $\Spc{X}$ to $\Spc{Y}$. Then, the \emph{natural discretization of $\Spc{M}$} is given by
 %
 \begin{equation*}
  \Discr_{m\times n}(\Spc{M}) = \left( \Spc{M}, \Field^{m\times n}, \REST{m\times n}{\Spc{M}}, \EXT{m\times n}{\Spc{M}}
  \right)
 \end{equation*}
 %
 with the operators
 %
 \begin{equation*}
  \REST{m\times n}{\Spc{M}}(\Op{T}) := \RmY \circ \Op{T} \circ \EnX
  \quad\text{and}\quad
  \EXT{m\times n}{\Spc{M}}(\DiscOp{T}) := \EmY \circ \DiscOp{T} \circ \RnX.
 \end{equation*}
 %
 We have here identified a linear operators from $\Field^n$ to $\Field^m$ with $\Field^{m\times n}$, the vector space 
 of $(m \times n)$ matrices with elements on $\Field$.
\end{definition}

\begin{remark}
 From the above considerations it becomes clear that it only makes sense to apply linear discretizations since 
 otherwise, the discretization of a linear operator is not necessary linear. \textbf{Hence, from now on, all 
 discretizations are assumed to be linear.}
\end{remark}


\subsubsection{Natural discretization of operator derivatives}

For $k \in \Natural$, the $k$'th Fr\'{e}chet derivative of a $k$ times continuously differentiable operator 
$\Op{T} \colon \Spc{X} \to \Spc{Y}$ is defined as a mapping
%
\begin{equation*}
 \partial^k \Op{T} \colon \Spc{X} \to \Spc{L}^k(\Spc{X}, \Spc{Y}),
\end{equation*}
%
where $\Spc{L}^k$ is recursively defined as 
%
\begin{equation*}
 \Spc{L}^0(\Spc{X}, \Spc{Y}) := \Spc{Y}, \quad 
 \Spc{L}^k(\Spc{X}, \Spc{Y}) := \Spc{L}\big( \Spc{X}, \Spc{L}^{k-1}(\Spc{X}, \Spc{Y}) \big).
\end{equation*}
%
This recursion can be resolved by setting
%
\begin{equation*}
 \bar\partial^k \Op{T} \colon \Spc{X} \times \Spc{X}^k \to \Spc{Y}
 \quad \text{with} \quad
 \bar\partial^k \Op{T}(f; h_1, \ldots, h_k) := \partial^k\Op{T}(f)(h_1)\cdots(h_k).
\end{equation*}
%
By definition, $\bar\partial^k\Op{T}$ is $k$-linear in the last $k$ arguments. We adopt this viewpoint in the following 
and write $\partial^k$ instead of $\bar\partial^k$.\\
%
The natural discretization of a product space as given in \Cref{def:product_space_discretization} induces a natural 
discretization of the $k$'th derivative similar to the natural operator discretization in 
\Cref{def:operator_discretization}.

\begin{definition}
 \label{def:derivative_discretization}
 Let $\Spc{X}$ and $\Spc{Y}$ be normed vector spaces with discretizations $\Discr_n(\Spc{X})$ and $\Discr_m(\Spc{Y})$, 
 respectively. Let further $\Op{T} \in \Spc{L}(\Spc{X}, \Spc{Y})$ be a $k$ times continuously Fr\'{e}chet 
 differentiable operator for some $k \in  \Natural$. The \emph{natural discretization of the $k$'th derivative} 
 $\partial^k \Op{T} \colon \Spc{X} \times \Spc{X}^k \to \Spc{Y}$ is defined as the mapping
 %
 \begin{equation*}
  \partial^k \DiscOp{T} \colon \Field^n \times \Field^{k\cdot n} \to \Field^m
  \quad \text{with} \quad
  \partial^k \DiscOp{T} = \RmY \circ \partial^k \Op{T} \circ (\EnX)^{k+1}.
 \end{equation*}
 %
 The notation $(\EnX)^{k+1}$ stands for the $(k+1)$-tuple $(\EnX, \ldots, \EnX)$, i.e., 
 %
 \begin{equation*}
  \partial^k \Op{T} \circ (\EnX)^{k+1}(\valpha; \vbeta_1, \ldots, \vbeta_k) :=
 \partial^k \Op{T}\big( \EnX(\valpha); \EnX(\vbeta_1), \ldots, \EnX(\vbeta_k) \big).
 \end{equation*}
\end{definition}


The following theorem shows that under rather weak conditions on the mappings $E_n^{\Spc{X}}$ and $R_m^{\Spc{Y}}$, the 
natural discretization of the operator derivative as in \Cref{def:derivative_discretization} is in fact the Fr\'{e}chet 
derivative of the natural operator discretization.

\begin{theorem}
 Let $\Spc{X}$ and $\Spc{Y}$ be normed vector spaces with discretizations as before, and let 
 $\Op{T} \in \Spc{L}(\Spc{X}, \Spc{Y})$ be a $k$ times continuously Fr\'{e}chet differentiable 
 operator. If $E_n^{\Spc{X}}$ and $R_m^{\Spc{Y}}$ are bounded, the natural discretization $\DiscOp{T}$ of $\Op{T}$ is 
 $k$ times  continuously Fr\'{e}chet differentiable, and its $k$'th Fr\'{e}chet derivative is the natural 
 discretization of $\partial^k\Op{T}$.
\end{theorem}

\begin{proof}
 We prove the theorem for $k=1$ since the statement for $k=0$ is trivial, and higher order derivatives follow by 
 induction.\\
 %
 For $\valpha,\vbeta \in \Field^n$, it is
 %
 \begin{equation*}
  \DiscOp{T}(\valpha + \vbeta) = \RmY \circ \Op{T} \circ \EnX (\alpha + \beta).
 \end{equation*}
 %
 From the definition of $\partial \Op{T}$ it follows that
 %
 \begin{align*}
  \Op{T}\big(\EnX(\valpha + \vbeta)\big)
  &= \Op{T}\big(\EnX(\valpha)\big) + \partial\Op{T}\big(\EnX(\valpha)\big)\big(\EnX(\vbeta)\big) +
  o\big(\NORMLR{\EnX(\vbeta)}_{\Spc{X}}\big) \\
  &= \Op{T} \circ \EnX(\valpha) + \partial\Op{T}\circ (\EnX)^2\,(\valpha; \vbeta) + o\big(\NORM{\vbeta}_n \big)
 \end{align*}
 %
 for $\NORM{\vbeta}_n \to 0$ since $\EnX$ is a bounded operator. By composing from left with the bounded operator 
 $\RmY$, one gets
 %
 \begin{align*}
  \DiscOp{T}(\valpha + \vbeta)
  &= \RmY \circ \Op{T} \circ \EnX(\valpha) + \RmY \circ \partial\Op{T}\circ (\EnX)^2\,(\valpha)(\vbeta) +
  o\big(\NORM{\vbeta}_n \big) 
  \\
  &= \DiscOp{T}(\alpha) + \RmY \circ \partial\Op{T}\circ (\EnX)^2\,(\valpha)(\vbeta) + o\big(\NORM{\vbeta}_n \big),
 \end{align*}
 %
 hence $\partial\DiscOp{T}$ is by definition given by the second term.
\end{proof}


\subsubsection{Natural discretization of the adjoint}

If $\Spc{X}$ and $\Spc{Y}$ are Hilbert spaces over the same field, one can consider the adjoint operator 
$\Op{T}^* \colon \Spc{Y} \to \Spc{X}$ and its natural discretization. The following theorem clarifies under which 
conditions this discretization is the adjoint of the operator discretization $\DiscOp{T}$.
%
\begin{lemma}
 Let $\Spc{X}$ and $\Spc{Y}$ are Hilbert spaces with linear discretizations as in \Cref{def:operator_discretization}, 
 and let $\Op{T} \colon \Spc{X} \to \Spc{Y}$ be a bounded linear operator. Then the natural discretization of 
 $\Op{T}^*$ given by
 %
 \begin{equation*}
  \DiscOp{T}^* \colon \Field^m \to \Field^n
  \quad \text{with} \quad
  \DiscOp{T}^* := \RnX \circ \Op{T}^* \circ \EmY
 \end{equation*}
 %
 is the adjoint of the natural discretization $\DiscOp{T} := \RmY \circ \Op{T} \circ \EnX$ of $\Op{T}$ if the operators 
 $\EnX$ and $\RmY$ are bounded and additionally satisfy
 %
 \begin{equation*}
  \EnX = (\RnX)^* \quad \text{and} \quad \EmY = (\RmY)^*.
 \end{equation*}
\end{lemma}


\begin{proof}
 Since $\EnX$ and $\RmY$ are bounded and linear, the same is true for $\DiscOp{T}$. Its adjoint is given by
 %
 \begin{equation*}
  \DiscOp{T}^* = (\RmY \circ \Op{T} \circ \EnX)^* = (\EnX)^* \circ \Op{T}^* \circ (\RmY)^*,
 \end{equation*}
 %
 which is apparently equal to the natural discretization $\RnX \circ \Op{T}^* \circ \EmY$ of $\Op{T}^*$ under the 
 stated conditions.
\end{proof}

\begin{remark}
 This statement can be generalized to densely defined linear operators since a unique adjoint exists also for this 
 class of operators. Another possible generalization is to dual pairings and instead of Hilbert spaces. The same rules 
 as above apply for these operators when chained with bounded linear operators from left or right.
\end{remark}


\subsubsection{Natural discretization of compositions}

Mappings or operators can be chained if their domains and ranges are compatible. We investigate how the natural 
discretization of an operator composition relates to the composition of the single discretized operators.

\begin{lemma}
 Let $\Spc{X}, \Spc{Y}$ and $\Spc{Z}$ be vector spaces with discretizations $\Discr_n(\Spc{X})$, 
 $\Discr_m(\Spc{Y})$ and $\Discr_k(\Spc{Z})$, respectively. Let further
 %
 \begin{equation*}
  \Op{T} \colon \Spc{X} \to \Spc{Y} \quad \text{and} \quad \Op{U} \colon \Spc{Y} \to \Spc{Z}
 \end{equation*}
 %
 be given operators. The natural discretization
 %
 \begin{equation*}
  \DiscOp{S} \colon \Field^n \to \Field^k 
  \quad \text{with} \quad
  \DiscOp{S} := \REST{k}{\Spc{Z}} \circ \Op{U} \circ \Op{T} \circ \EnX
 \end{equation*}
 %
 of the operator composition $\Op{U} \circ \Op{T}$ is equal to the composition of the discretized operators
 %
 \begin{equation*}
  \DiscOp{T} := \RmY \circ \Op{T} \circ \EnX
  \quad \text{and} \quad
  \DiscOp{U} = \REST{k}{\Spc{Z}} \Op{U} \circ \EmY
 \end{equation*}
 %
 if either the operator $\PmY = \EmY \circ \RmY$ is the identity on the range of $\Op{T} \circ \EnX$ or if for each 
$\vnu \in \Field^k$,  
 the preimage $(\REST{k}{\Spc{Z}} \Op{U} \circ \EmY)^{-1}(\{\vnu\})$ is invariant under $\PmY$.
\end{lemma}

\begin{proof}
 Obviously, it is
 %
 \begin{equation*}
  \DiscOp{U} \circ \DiscOp{T} = \REST{k}{\Spc{Z}} \Op{U} \circ \EmY \circ \RmY \circ \Op{T} \circ \EnX,
 \end{equation*}
 %
 which is equal to $\DiscOp{S}$ as defined above if the factor $\PmY = \EmY \circ \RmY$ can be dropped. This is the 
 case under the given condition as shown in the proof of \Cref{lemma:nat_op_discr_fulfills_corresp}.
\end{proof}


\subsection{Compatibility of operators under discretizations}

We proceed by investigating conditions that a discretization should fulfill. Often, we need to perform various 
operations on $\Spc{X}$, and there are corresponding operations on $\Field^{n}$. It is then important that such 
operations are compatible \wrt the discretization. In the general setting, most of these operations can be stated as 
mappings between appropriately chosen vector spaces. Thus, we begin with defining the notion of compatibility of 
mappings \wrt discretizations. 
%
\begin{definition}
 \label{def:operator_compatibility}
 Let $\Spc{X}$ and $\Spc{Y}$ be vector spaces over $\Field$ with discretizations $\Discr_{n}(\Spc{X})$ and 
 $\Discr_{m}(\Spc{Y})$, respectively. Now, consider the operators
 %
 \begin{equation*}
  \Op{T} \colon \Spc{X} \to \Spc{Y} \quad\text{and}\quad \DiscOp{T} \colon \Field^{n} \to \Field^{m}.
 \end{equation*}
 \vspace{-2\baselineskip}
 \begin{enumerate}[(a)]
  \item \label{def:operator_compatibility:a_exact}
  \emph{$\DiscOp{T}$ exactly corresponds to $\Op{T}$ under the discretizations $\Discr_{n}(\Spc{X})$ and  
  $\Discr_{m}(\Spc{Y})$} if the following holds for any $n,m \in \Natural$:
  %
  \begin{equation*}
   \RmY \circ \Op{T} = \DiscOp{T} \circ \RnX  \quad\text{and}\quad  \EmY \circ \DiscOp{T} = \Op{T} \circ \EnX.
  \end{equation*}

  \item \label{def:operator_compatibility:b_approximately}
  \emph{$\DiscOp{T}$ approximately corresponds to $\Op{T}$ under the discretizations $\Discr_{n}(\Spc{X})$ and 
  $\Discr_{m}(\Spc{Y})$} if the following holds:
  %
  \begin{align*} 
   \lim_{n,m\to \infty} \NORMLR{\RmY \circ \Op{T} - \DiscOp{T} \circ \RnX}_{m} &= 0  \\[0.5em]
   \lim_{n,m\to \infty} \NORMLR{\EmY \circ \DiscOp{T} - \Op{T} \circ \EnX}_{\Spc{Y}} &= 0.
  \end{align*}
  %
 \end{enumerate}  
 If any of the above holds, then we say that $\DiscOp{T}$ and $\Op{T}$ \emph{correspond to each other under the 
 discretizations $\Discr_{n}(\Spc{X})$ and $\Discr_{m}(\Spc{Y})$}.
\end{definition}

In the following Lemma, we investigate under which conditions the natural discretization of an operator leads to 
compatibility in the sense of the above definition. These conditions can be stated more specifically in special cases, 
see ??

\begin{lemma}
 \label{lemma:nat_op_discr_fulfills_corresp}
 Let $\DiscOp{T}$ be the natural linear discretization of an operator $\Op{T}$ based on linear space discretizations as 
 in \Cref{def:operator_discretization}, and let $\PnX = \EnX \circ \RnX$ and $\PmY = \EmY \circ \RmY$ as in 
 \Cref{def:space_discretization}. Then the following holds:
 
 \begin{enumerate}[(a)]
  \item \label{lemma:nat_op_discr_fulfills_corresp:a_general}
  $\DiscOp{T}$ and $\Op{T}$ exactly correspond to each other if and only if (1) for each $\vbeta \in \Field^m$, the 
  preimage $(\RmY \circ \Op{T})^{-1}(\{\vbeta\})$ is invariant under $\PnX$ and (2) $\PmY$ is the identity operator on 
  the range of $\Op{T} \circ \EnX$.

  \item \label{lemma:nat_op_discr_fulfills_corresp:b_hilbert_linear}
  If $\Spc{X}$ is a Hilbert space and $\Op{T}$ and $\RmY$ are linear and bounded, the condition on $\PnX$ in 
  \eqref{lemma:nat_op_discr_fulfills_corresp:a_general} can be rephrased as the following: (1a) $\PnX$ is equal to the 
  identity operator on $\nullspace(\RmY \circ \Op{T})^\perp$ and (1b) $\nullspace(\RmY \circ \Op{T})$ is invariant 
  under $\PnX$.
  
%  \item Something about approximate correspondence.
 \end{enumerate}
\end{lemma}


\begin{proof}
 \begin{enumerate}[a)]
  \item Right-composing the discretized operator $\DiscOp{T} = \RmY \circ \Op{T} \circ \EnX$ with $\RnX$ results in
  %
  \begin{equation*}
   \DiscOp{T} \circ \RnX = \RmY \circ \Op{T} \circ \PnX,
  \end{equation*}
  %
  and by left-composing $\DiscOp{T}$ with $\EmY$, we get
  %
  \begin{equation*}
   \EmY \circ \DiscOp{T} = \PmY \circ \Op{T} \circ \EnX.
  \end{equation*}
  %
  Let now $\vbeta \in \Field^m$ be arbitrary and $A := (\RmY \circ \Op{T})^{-1}(\{\vbeta\})$ be its preimage.
  To show the ``if'' part, we observe that for each $f \in A$, it is $\PnX(f) \in A$ since $A$ is invariant under 
  $\PnX$. Hence, $\RmY \circ \Op{T} \big( \PnX(f) \big) = \vbeta$. This shows 
  $\RmY \circ \Op{T} \circ \PnX = \RmY \circ \Op{T}$, i.e. the first condition 
  in \Cref{def:operator_compatibility}~\eqref{def:operator_compatibility:a_exact} is fulfilled.\\
  %
  Obviously, if $\PmY$ is the identity on the range of $\Op{T} \circ \EnX$, the right hand side of the second identity 
  is $\PmY \circ \Op{T} \circ \EnX = \Op{T} \circ \EnX$. This implies the second equation in 
  \Cref{def:operator_compatibility}~\eqref{def:operator_compatibility:a_exact} and concludes this part of the proof.\\
  %
  For the ``only if'' part, we assume that there is $f \in A$ with $\PnX(f) \not\in A$. By definition, this means 
  $\RmY \circ \Op{T} \big( \PnX(f) \big) \neq \vbeta$, which violates the condition
  $\vbeta = \RmY \circ \Op{T} (f) = \RmY \circ \Op{T} \circ \PnX (f)$. Hence, $A$ must be invariant under $\PnX$.\\
  %
  Apparently, $\PmY \circ \Op{T} \circ \EnX = \Op{T} \circ \EnX$ means that for each $g = \Op{T} \circ \EnX(\valpha)$, 
  it holds $\PmY(g) = g$, i.e. $\PmY$ is the identity on the range of $\Op{T} \circ \EnX$.

  \item To prove \eqref{lemma:nat_op_discr_fulfills_corresp:b_hilbert_linear}, we first observe that 
  $\nullspace(\RmY \circ \Op{T}) = (\RmY \circ \Op{T})^{-1}(\{\vzero\})$. Further, since $\Spc{X}$ is a Hilbert space 
  and $\RmY \circ \Op{T}$ linear and bounded, each $f \in \Spc{X}$ can be uniquely decomposed into 
  $f = f_0 + \tilde f$ with  $f_0 \in \nullspace(\RmY \circ \Op{T})$ and 
  $\tilde f \in \nullspace(\RmY \circ \Op{T})^\perp$. Hence, for each $\vbeta \in \Field^m$,
  %
  \begin{equation*}
   (\RmY \circ \Op{T})^{-1}({\vbeta}) = \tilde f + \nullspace(\RmY \circ \Op{T})
  \end{equation*}
  %
  for a unique $\tilde f \in \nullspace(\RmY \circ \Op{T})^\perp$. Thus, the conditions on $\PnX$ in 
  \eqref{lemma:nat_op_discr_fulfills_corresp:b_hilbert_linear} are equivalent to those in   
  \eqref{lemma:nat_op_discr_fulfills_corresp:a_general}.
  
%  \item 
 \end{enumerate}
\end{proof}



\subsubsection{Compatibility of linear combinations in vector spaces}

Let $\Spc{X}$ be a vector space over $\Field$ with its usual discretization. The linear combination in vector spaces 
can be formally defined as an operator, and as such it implies a natural discretization. In the following, we 
investigate how it relates to the usual linear combination in $\Field^n$ and under which conditions they correspond to 
each other.

\begin{lemma}
 Let $\Spc{X}$ be a vector space over $\Field$ with linear discretization $\Discr_{n}(\Spc{X})$.
 \begin{enumerate}[(a)]
  \item The linear combination in $\Field^n$ is the natural discretization of the linear combination in $\Spc{X}$ if 
  and only if the projection $\QnX = \RnX \circ \EnX$ is the identity.
  
  \item The linear combinations in $\Spc{X}$ and $\Field^n$ exactly correspond to each other.
 \end{enumerate}
\end{lemma}

\begin{proof}
 \begin{enumerate}[(a)]
  \item The linear combination in $\Spc{X}$ can be written as an operator 
  %
  \begin{equation*}
   \Op{T} \colon \Field \times \Field \times \Spc{X} \times \Spc{X} \to \Spc{X}
   \quad \text{with} \quad
   \Op{T}(\mu, \lambda, f, g) := \lambda \cdot f + \mu \cdot g
  \end{equation*}
  %
  with natural discretization $\DiscOp{T} = \RnX \circ \Op{T} \circ (\Op{\Id}, \Op{\Id}, \EnX, \EnX)$.
  Hence, for $\lambda, \mu \in \Field$ and $\valpha, \vbeta \in \Field^n$, it is
  %
  \begin{align*}
   \DiscOp{T}(\lambda, \mu, \valpha, \vbeta) 
   &= \RnX \circ \Op{T} \big( \lambda, \mu, \EnX(\valpha), \EnX(\vbeta) \big)\\
   &= \RnX \big( \lambda \cdot \EnX(\valpha) + \mu \cdot \EnX(\vbeta) \big) \\ 
   &= \RnX \circ \EnX(\lambda \cdot \valpha + \mu \cdot \vbeta).
  \end{align*}
  %
  This expression is equal to $\lambda \cdot \valpha + \mu \cdot \vbeta$ for all $\lambda, \mu \in \Field$ and all 
  $\valpha, \vbeta \in \Field^n$ if and only if $\RnX \circ \EnX = \Op{\Id}$.

  \item The conditions for exact correspondence according to \Cref{def:operator_compatibility} can be rephrased as
  %
  \begin{align*}
   \RnX(\lambda \cdot f + \mu \cdot g) &= \lambda \cdot \RnX(f) + \mu \cdot \RnX(g), \\
   \EnX(\lambda \cdot \valpha + \mu \cdot \vbeta) &= \lambda \cdot \EnX(\valpha) + \mu \cdot \EnX(\vbeta),
  \end{align*}
  %
  and are fulfilled due to the linearity of $\RnX$ and $\EnX$.
 \end{enumerate}
\end{proof}

\begin{remark}
 In the vector space $\Spc{M} := \Spc{L}(\Spc{X}, \Spc{Y})$ with given discretizations in $\Spc{X}$ and $\Spc{Y}$, the 
 natural restriction and extension operators of the discretization $\Discr_{m\times n}(\Spc{M})$ are given by
 %
 \begin{equation*}
  \REST{m\times n}{\Spc{M}}(\Op{T}) = \RmY \circ \Op{T} \circ \EnX
  \quad\text{and}\quad
  \EXT{m\times n}{\Spc{M}}(\DiscOp{T}) = \EmY \circ \DiscOp{T} \circ \RnX,
 \end{equation*}
 %
 see \Cref{def:operator_space_discretization}. Hence, the condition that the natural discretization of the linear 
 combination in $\Spc{M}$ is the linear combination in $\Field^{m\times n}$ reads as
 %
 \begin{align*}
  \DiscOp{T} 
  &= \REST{m\times n}{\Spc{M}}\big( \EXT{m\times n}{\Spc{M}}(\DiscOp{T}) \big)
  = \RmY \circ \EXT{m\times n}{\Spc{M}}(\DiscOp{T}) \circ \EnX
  = \RmY \circ \EmY \circ \DiscOp{T} \circ \RnX \circ \EnX \\
  &= \QmY \circ \DiscOp{T} \circ \QnX.
 \end{align*}
 %
 Thus, if $\QnX = \Op{\Id}$ and $\QmY = \Op{\Id}$, the operator space discretization $\Discr_{m\times n}(\Spc{M})$ 
 automatically fulfills the condition $\COPROJ{m\times n}{\Spc{M}} = \Op{\Id}$ needed for the natural discretization of 
 the linear combination.
\end{remark}



\subsection{Compatibility of norms}

We now consider the setting when $\Spc{X}$ is a normed vector space, including the space of bounded linear operators.

\begin{lemma}
 \label{lemma:norm_compatibility}
 Let $\Spc{X}$ be a vector space over $\Field$ with norm $\NORM{\cdot}_{\Spc{X}}$ and linear discretization 
 $\Discr_n(\Spc{X})$. Let further $\NORM{\cdot}_n$ be a norm in $\Field^n$.
 %
 \begin{enumerate}[(a)]
  \item \label{lemma:norm_compatibility:a_natural}
  $\NORM{\cdot}_n$ is the natural discretization of $\NORM{\cdot}_{\Spc{X}}$ if and only if
  %
  \begin{equation*}
   \EnX \colon (\Field^n, \NORM{\cdot}_n) \to (\Spc{X}, \NORM{\cdot}_{\Spc{X}})
  \end{equation*}
  %
  is an isometry.
  
  \item \label{lemma:norm_compatibility:b_correspondence}
  $\NORM{\cdot}_n$ corresponds to $\NORM{\cdot}_{\Spc{X}}$ if and only if
  %
  \begin{equation*}
   \RnX \colon (\Spc{X}, \NORM{\cdot}_{\Spc{X}}) \to (\Field^n, \NORM{\cdot}_n)
   \quad \text{and} \quad
   \EnX \colon (\Field^n, \NORM{\cdot}_n) \to (\Spc{X}, \NORM{\cdot}_{\Spc{X}})
  \end{equation*}
  %
  are isometries.
 \end{enumerate}
\end{lemma}

\begin{proof}
 We write the norm as an operator
 %
 \begin{equation*}
  \Op{T} \colon \Spc{X} \to \Real \quad \text{with} \quad \Op{T}(f) := \NORM{f}_{\Spc{X}}.
 \end{equation*}
 %
 Its natural discretization is given by
 %
 \begin{equation*}
  \DiscOp{T} \colon \Field^n \to \Real \quad \text{with} \quad \DiscOp{T} = \Op{\Id} \circ \Op{T} \circ \EnX,
 \end{equation*}
 %
 \ie, $\DiscOp{T}(\valpha) = \NORM{\EnX(\valpha)}_{\Spc{X}}$. This proves the claim in 
 \eqref{lemma:norm_compatibility:a_natural}.\\
 %
 The conditions for correspondence read as
 %
 \begin{equation*}
  \NORM{\cdot}_{\Spc{X}} = \NORM{\cdot}_n \circ \RnX
  \quad \text{and} \quad
  \NORM{\cdot}_n = \NORM{\cdot}_{\Spc{X}} \circ \EnX
 \end{equation*}
 %
 and are equivalent to the isometry conditions in \eqref{lemma:norm_compatibility:b_correspondence}.
\end{proof}

\begin{remark}
 The proof additionally shows that the natural norm discretization fulfills
 $\DiscOp{T}(\valpha) \leq \NORM{\EnX}\, \NORM{\valpha}_n$, i.e. the discretized norm can be bounded from above 
 with the given norm in $\Field^n$.
\end{remark}


\paragraph{Operator norms}

In a similar way, we can analyze the relation between norms of continuous and discretized linear operators. 
The norms of operators $\Op{T} \in \Spc{M} := \Spc{L}(\Spc{X}, \Spc{Y})$ and 
$\DiscOp{T} \in \Spc{L}(\Field^n, \Field^m) \simeq \Field^{m\times n}$ are defined as
%
\begin{equation*}
 \NORM{\Op{T}}_{\Spc{M}} := \sup_{\NORM{f}_{\Spc{X}} \leq 1} \NORM{\Op{T}(f)}_{\Spc{Y}}
 \quad \text{and} \quad
 \NORM{\DiscOp{T}}_{m\times n} := \sup_{\NORM{\valpha}_n \leq 1} \NORM{\DiscOp{T}(\valpha)}_m.
\end{equation*}
%
The following Lemma shows how operator and matrix norms relate to each other under discretization.

\begin{lemma}
 \label{lemma:operator_norm_estimate}
 Let $\Spc{X}$ and $\Spc{Y}$ be normed vector spaces with linear discretizations $\Discr_n(\Spc{X})$ and 
 $\Discr_m(\Spc{Y})$, respectively. Let further $\REST{m\times n}{\Spc{M}}(\Op{T})$ and 
 $\EXT{m\times n}{\Spc{M}}(\DiscOp{T})$ be the natural restriction and extension operators in 
 $\Discr_{m\times n}(\Spc{M})$ as given in \Cref{def:operator_space_discretization}.
 %
 \begin{enumerate}[(a)]
  \item \label{lemma:operator_norm_estimate:a_rest}
  If the operator norms of $\RmY$ and $\EnX$ are bounded by 1, then
  %
  \begin{equation*}
   \NORM{R_{m\times n}^{\Spc{M}}}_{\Spc{L}(\Spc{M},\Field^{m\times n})} \leq 1.
  \end{equation*}
  %
  \item \label{lemma:operator_norm_estimate:a_rest}
  If the operator norms of $\RnX$ and $\EmY$ are bounded by 1, then
  %
  \begin{equation*}
   \NORM{E_{m\times n}^{\Spc{M}}}_{\Spc{L}(\Field^{m\times n}, \Spc{M})} \leq 1.
  \end{equation*}
  %
  Furthermore, if $\RnX$ and $\EmY$ are isometries and $\RnX$ is surjective, $E_{m\times n}^{\Spc{M}}$ is an isometry, 
  \ie, $\NORM{\cdot}_{m\times n}$ is the natural discretization of $\NORM{\cdot}_{\Spc{M}}$.
 \end{enumerate}
\end{lemma}


\begin{proof}
 \begin{enumerate}[(a)]
  \item For $\Op{T} \in \Spc{M}$ and $\DiscOp{T} = \REST{m\times n}{\Spc{M}}(\Op{T})$ we have
  %
  \begin{equation*}
   \NORMLR{\DiscOp{T}(\valpha)}_m = \NORMLR{\RmY \left( \Op{T}\big( \EnX(\valpha) \big) \right)}_m
   \leq \NORMLR{\Op{T}\big( \EnX(\valpha) \big)}_{\Spc{Y}}
  \end{equation*}
  %
  since $\NORM{\RmY} \leq 1$. Furthermore, $\EnX$ maps the unit ball in $\Field^n$ to a subset of the unit ball in
  $\Spc{X}$ since $\NORM{\EnX} \leq 1$. Hence,
  %
  \begin{equation*}
   \sup_{\NORM{\valpha}_n \leq 1} \NORMLR{\Op{T}\big( \EnX(\valpha) \big)}_{\Spc{Y}}
   = \sup_{\STACK{f=\EnX(\valpha)}{\NORM{\valpha}_n \leq 1}} \NORMLR{\Op{T}(f)}_{\Spc{Y}}
   \leq \sup_{\NORM{f}_{\Spc{X}} \leq 1} \NORMLR{\Op{T}(f)}_{\Spc{Y}},
  \end{equation*}
  %
  which proves that $\NORMLR{R_{m\times n}^{\Spc{M}}(\Op{T})} \leq \NORM{\Op{T}}$.

  \item Analogously, for $\DiscOp{T} \in \Field^{m\times n}$ and $\Op{T} = \EXT{m\times n}{\Spc{M}}(\DiscOp{T})$ it is
  %
  \begin{equation*}
   \NORMLR{\Op{T}(f)}_{\Spc{Y}} = \NORMLR{\EmY \left( \DiscOp{T}\big( \RnX(f) \big) \right)}_{\Spc{Y}}
   \leq \NORMLR{\DiscOp{T}\big( \RnX(f) \big)}_m
  \end{equation*}
  %
  since $\NORM{\EmY} \leq 1$. In addition, $\RnX$ maps the unit ball in $\Spc{X}$ to a subset of the unit 
  ball in $\Field^n$ since $\NORM{\RnX} \leq 1$. Hence,
  %
  \begin{equation*}
   \sup_{\NORM{f}_{\Spc{X}} \leq 1} \NORMLR{\DiscOp{T}\big( \RnX(f) \big)}_m
   = \sup_{\STACK{\valpha=\RnX(f)}{\NORM{f}_{\Spc{X}} \leq 1}} \NORMLR{\DiscOp{T}(\valpha)}_m
   \leq \sup_{\NORM{\valpha}_n \leq 1} \NORMLR{\DiscOp{T}(\valpha)}_m,
  \end{equation*}
  %
  \ie, $\NORMLR{E_{m\times n}^{\Spc{M}}(\DiscOp{T})} \leq \NORM{\DiscOp{T}}$.\\[1em]
  %
  If $\EmY$ is an isometry, the first inequality becomes an equality, and if $\RnX$ is a surjective isometry, it maps
  the unit ball in $\Spc{X}$ \emph{onto} the unit ball in $\Field^n$. Thus, also the second inequality becomes an 
  equality, showing that $\NORMLR{E_{m\times n}^{\Spc{M}}(\DiscOp{T})} = \NORM{\DiscOp{T}}$ for all 
  $\DiscOp{T} \in \Field^{m\times n}$ in this case.
 \end{enumerate}
\end{proof}


\begin{remark}
 An analogous isometry property for $\REST{m\times n}{\Spc{M}}$ can only hold if $\DIM(\Spc{X}) \leq n$ 
 since it requires the extension map $\EnX \colon \Field^n \to \Spc{X}$ to be surjective. Hence, exact correspondence 
 of a matrix norm to an operator norm is only possible for certain finite-dimensional spaces $\Spc{X}$.\\
 %
 In any case, the inequalities ensure that one can get an estimate for the operator norm of $\Op{T}$ (restricted to the 
 range of $\EnX$) in terms of the discretized (matrix) norm of $\DiscOp{T}$, which can be very useful in numerical 
 computations.
\end{remark}


\subsection{Compatibility of inner products}

Now we consider inner product spaces and analyze their behavior under discretization.

\begin{lemma}
 \label{lemma:inner_prod_compatibility}
 Let $(\Spc{X}, \INNER{\cdot}{\cdot}_{\Spc{X}})$ be an inner product space with discretization $\Discr_n(\Spc{X})$, and 
 let $\INNER{\cdot}{\cdot}_n$ be the inner product in $\Field^n$.
 %
 \begin{enumerate}[(a)]
  \item \label{lemma:inner_prod_compatibility:a_natural}
  $\INNER{\cdot}{\cdot}_n$ is the natural discretization of the inner product $\INNER{\cdot}{\cdot}_{\Spc{X}}$ if and 
  only if $\EnX$ is unitary.
  
  \item \label{lemma:inner_prod_compatibility:b_correspondence}
  $\INNER{\cdot}{\cdot}_n$ exactly corresponds to $\INNER{\cdot}{\cdot}_{\Spc{X}}$ if and only if both $\RnX$ and 
  $\EnX$ are unitary.
 \end{enumerate}
\end{lemma}

\begin{proof}
 \begin{enumerate}[(a)]
  \item We define the inner product as an operator
  %
  \begin{equation*}
   \Op{T} \colon \Spc{X} \times \Spc{X} \to \Field
   \quad \text{with} \quad
   \Op{T}(f, g) := \INNER{f}{g}_{\Spc{X}}.
  \end{equation*}
  %
  The natural discretization of this operator is given by $\DiscOp{T} = \Op{\Id} \circ \Op{T} \circ (\EnX, \EnX)$, \ie,
  %
  \begin{equation*}
   \DiscOp{T}(\valpha, \vbeta) 
   = \Op{T}\big( \EnX(\valpha), \EnX(\vbeta) \big) 
   = \INNERbig{\EnX(\valpha)}{\EnX(\vbeta)}_{\Spc{X}}
   = \INNERbig{(\EnX)^* \circ \EnX(\valpha)}{\vbeta}_n.
  \end{equation*}
  %
  Apparently, this expression is equal to $\INNER{\valpha}{\vbeta}_n$ if and only if $\EnX$ is unitary.

  \item The conditions for exact correspondence can be written as
  %
  \begin{equation*}
   \INNER{f}{g}_{\Spc{X}} = \INNERbig{\RnX(f)}{\RnX(g)}_n
   \quad \text{and} \quad
   \INNER{\valpha}{\vbeta}_n = \INNERbig{\EnX(\valpha)}{\EnX(\vbeta)}_{\Spc{X}},
  \end{equation*}
  %
  which is equivalent to the conditions in the claim.
 \end{enumerate}
\end{proof}


\section{Specific discretizations}
\label{sec:specific_discretizations}

We will now consider a specific family of discretizations of a Hilbert space $\Spc{X}$ over the 
field $\Field$ which contains all the cases we are interested in. 
\begin{definition}
  Let $V:=\{ \phi_{1},\ldots,\phi_{m} \}\subset \Spc{X}$ be a fixed finite subset of $m$ elements in $\Spc{X}$.
  Then we make the following definitions:
  \begin{enumerate}
  \item The \emph{restriction operator $R_{m}(\cdot;V)$ induced by $V$} is the  mapping  
    \[ R_{m}(\cdot;V) \colon H \to \Field^{m} \]
    where 
    \[ R_{m}(f;V):=\bigl( r_{1}(f;V),\ldots,r_{m}(f;V) \bigr) 
       \quad\text{for all $f\in \Spc{X}$,} \]
    with $r_{i}(f;V)\in \Field$ for $i=1,\ldots,m$.
  \item The \emph{extension operator $E_{m}(\cdot;V)$ induced by 
    $V$} is defined as the mapping  
    \[ E_{m}(\cdot;V) \colon \Field^{m} \to \Spc{X} \]
    where 
    \[ E_{m}(\beta;V):=  \sum_{i=1}^m \beta_i \phi_{i}
       \quad\text{for all $\beta\in \Field^{m}$.} \]      
  \end{enumerate}
\end{definition}


We do not further specify how $r_{i}(\cdot;V)$ is defined.
    Simply note that two choices of $r_{i}(\cdot;V)$ are common, namely
    $r_{i}(f;V):=f(x_{i})$ where $x_{i}\in \Spc{X}$ is a fixed point (a sample 
    point) and $r_{i}(f;V):=\ip{f}{\phi_{i}}_{H}$ which is natural whenever
    $V$ is a subset of an orthonormal basis in $H$. Also, 
    $r_{i}(\cdot;V)$ is linear in both these cases.
    In this context the elements in $V$ are called the 
    \emph{sampling functions}.

Note that the sets of sampling and extrapolation functions do not 
need to coincide.  The only requirement is that the number of 
elements $m$ must be the same. Thus, a pair 
$(V^m_{\rest},V^m_{\ext})$ of $m$ 
sampling and extrapolation functions defines an $m$-dimensional 
discretization of $H$ by the tuple
\[ \bigl\{
     H,Z^m,R_{m}(\cdot;V^m_{\rest}),E_{m}(\cdot;V^m_{\ext})
   \bigr\}. \]
   
Now, consider two the following two pairs $(V^n_{\rest},V^n_{\ext})$ and 
$(V^m_{\rest},V^m_{\ext})$ of $n$ and $m$ sampling and extrapolation 
functions:
\begin{alignat*}{2}
  V^n_{\rest} &:=\{ \phi_{1},\ldots,\phi_{n} \} &\qquad
  V^m_{\rest} &:=\{ \psi_{1},\ldots,\phi_{m} \}  \\
  V^n_{\ext} &:=\{ \wt{\phi}_{1},\ldots,\wt{\phi}_{n} \} &\qquad
  V^m_{\ext} &:=\{ \wt{\psi}_{1},\ldots,\wt{\psi}_{m} \}    
\end{alignat*}
Each pair yields a discretization of $\cF_{\Field}(\Spc{X},Z)$ by the tuples
\[ 
   \bigl\{
     \cF_{\Field}(\Spc{X},Z),Z^n,\RnX(\cdot;V^n_{\rest}),\EnX(\cdot;V^n_{\ext})
   \bigr\} \]
and
\[ \bigl\{
     \cF_{\Field}(\Spc{X},Z),Z^m,R_{m}(\cdot;V^m_{\rest}),E_{m}(\cdot;V^m_{\ext})
   \bigr\}. \] 
What is the relation between these two discretizations? A function 
$f\in \cF_{\Field}(\Spc{X},Z)$ is represented by the ``vectors'' 
\[ z_{n}:= \RnX(f;V^n_{\rest})\in Z^n \quad\text{and}\quad
   z_{m}:= R_{m}(\cdot;V^m_{\rest})\in Z^m \]
and we seek the relaation between $z_{n}$ and $z_{m}$. 
Now making use of the approximation $f\approx 
\EnX(z_{n};V^n_{\ext})$, \ie,
\[ f \approx x \mapsto 
       \sum_{i=1}^n z_{n,i} \wt{\phi}_{i}(x)
   \quad\text{in $\cF_{\Field}(\Spc{X},Z)$,} 
\]
which implies that
\begin{align*}
  z_{m} &:=R_{m}(f;V^m_{\rest}) =
    \bigl( r_{1}(f,V^m_{\rest}),\ldots,r_{m}(f,V^m_{\rest}) \bigr) \\
    &\approx 
      \Bigl( r_{1}\bigl(\EnX(z_{n},V^n_{\ext}),V^m_{\rest}\bigr),
             \ldots,
	     r_{m}\bigl(\EnX(z_{n},V^n_{\ext}),V^m_{\rest}\bigr)
      \Bigr).
\end{align*}
When $r_{k}(\cdot,V^m_{\rest})$ are linear, then 
\[ r_{k}\bigl(\EnX(z_{n},V^n_{\ext}),V^m_{\rest}\bigr)=
     r_{k}\biggl(\sum_{i=1}^n 
       z_{n,i}\wt{\phi}_{i}(\cdot),V^m_{\rest}\biggr)=
     \sum_{i=1}^n z_{n,i} r_{k}(\wt{\phi}_{i},V^m_{\rest}).
\]   
Thus, for $k=1,\ldots,m$
\[
  z_{m,k}\approx 
    \sum_{i=1}^n z_{n,i} r_{k}(\wt{\phi}_{i},V^m_{\rest}).
\]    
and defining 
\[ A_{i,k}:= r_{k}(\wt{\phi}_{i},V^m_{\rest})
   \text{ for $i=1,\ldots,n$ and $j=1,\ldots,m$,} \]
enables us to write
\[ z_{m}=z_{n}\cdot\vA
   \quad\text{where } 
   \vA:=[ A_{i,k} ]_{\substack{1\leq i \leq n \\ 1\leq k \leq m}}
   \text{ is an $(n\times m)$-matrix.} \]
Conversly, defining 
\[ \wt{A}_{k,i}:=r_{i}(\phi_{k},V^n_{\rest})
   \text{ for $i=1,\ldots,n$ and $k=1,\ldots,m$,} \]
enables us to write
\[ z_{n}=z_{m}\cdot\wt{\vA}
   \quad\text{where } 
   \wt{\vA}:=[ \wt{A}_{i,k} ]_{\substack{1\leq k \leq M \\ 
   1\leq i \leq n}}
   \text{ is an $(m\times n)$-matrix.} \]
Finally, if $V^n_{\rest}=V^n_{\rest}$ and 
$V^m_{\ext}=V^m_{\rest}$, then $\vA=\wt{\vA}^T$. 
%
%
%\paragraph{Discretization of functions spaces.}
%We begin by introducing some useful notation that well be used 
%throughout this section.
%\begin{Not}
%  Let $\Spc{X}$ be a topological space, $\mu$ a fixed 
%  Borel measure, $\Field$ a field, and $Z$ a vector space over $\Field$. 
%  Then we define $\cL^1_{\loc,\Field}(\Spc{X},Z)$ as the Hilbert space over $\Field$ of
%  locally $\mu$-measurable functions 
%  $f\colon \Spc{X}\to Z$. The notation $\cL^1(\domain)$ simply stands for 
%  $\cL^1_{\loc,\Real}(\domain,\Real)$.
%\end{Not}  
%Throughout this section $Z$ and $W$ are vector spaces over the 
%fields $\Field$ and $\Field'$ and $\Spc{X}, \Spc{Y}$ are topological spaces.
%Also, let
%\[ \Spc{X}:=\cF_{\Field}(\domain,Z)\subset \cL^1_{\loc,\Field}(\domain,Z) \]
%denote a  Hilbert subspace.   
%
%
%\section{Relation to inverse problems}
%Let $\Spc{X}, \Spc{Y}$ be topological spaces and $T \colon \cF(X,\Field) \to 
%\cF(\Spc{Y},\Field')$. The inverse problem we are interested in is to find 
%$f\in H_{\Spc{X}}:=\cF(X,\Field)$ that solves the operator equation $T(f)=g$ where 
%$g\in H_{\Spc{Y}}:=\cF(Y,\Field')$ is 
%known only on a subset $Y_{0}\subset Y$. In reality $Y_{0}$ is a 
%finite set, so $g$ is known only at a finite number of points in $Y$. 
%
%This enables us to replace the continuous inverse  is given, with a sequence of 
%discrete inverse problems of finding $f_{n}\in H_{n}$ such that 
%$\DiscOp{T}(f_{n})=R_{m}(g)$.
%
%Finally, note that the dimension of the discretization of  $\cF(Y,W)$ 
%occurring in the inverse problem in subsection~\ref{sec:Inv} is naturally 
%determined by the fact that $T(f_{\true})$ is known (up to measurement 
%errors) on a given set $Y_{0}\subset Y$ with $m$ elements. 
%It is natural to choose an $m$-dimensional discretization of 
%$\cF(Y,W)$ where the choice of the sampling and extrapolation functions are 
%related to the elements in $Y_{0}$.
%
%
%
%   
%\section{The object model}
%What we are looking for is an efficient object structure representing 
%the discretization problem described above. The idea is that given an 
%operator 
%\[ T \colon \cF(X) \to \cF(Y) \]
%the object structure in the code must represent the discretization to 
%the level of accuracy that enables one to change between different
%discretizations $\DiscOp{T}_{N,M}$ and $\DiscOp{T}_{N',M'}$ of $T$, at least 
%for different sample based discretizations.
%
%The naive way to model the object structure is to fully replicate the 
%mathematical setting (fields, vector spaces over fields, normed spaces, Hilbert 
%spaces over fields, etc) but this produces a very large object structure 
%that we believe is not feasible nor necessary. The idea that we have 
%played with
%is to have an object structure that is general enough to capture different 
%choices of sample based discretizations and relations in between them 
%without being inefficient. 
%As an example, what possible function spaces $\cF(X)$ do we have? Well, 
%mathematically this depends on $T$ and examples for $X=\Real^n$ 
%are $\Cc_{c}(\Real^n)$ (continuous functions with compact 
%support), $\Cc^\infty_{c}(\Real^n)$ (smooth functions with compact 
%support), and 
%$\mathcal{S}(\Real^n)$ (functions in the Schwartz class, \ie, fast 
%decaying smooth functions). When using 
%a sample based discretization it is actually not relevant exactly which 
%Hilbert space that one started with. The only important factor is the 
%underlying field and the inner product (which affects the inner 
%product in the finite-dimensional representation) since the same 
%sample based discretization yields the same finite-dimensional representation for 
%all Hilbert spaces over the same field with the same inner product structure.
%As an example a fixed sample based discretization can not differ between bounded 
%integrable functions or 
%continuous functions if the inner product that we use is the same. 
%Hence, a Hilbert space may be represented by the underlying field 
%($\Real$, $\Complex$,\ldots), and the inner product structure (or at least its 
%discrete counterpart).
%
%\subsection{Consistency checks}
%It is desirable that the object model can check for consisteny. The 
%first problem is to be able to compare two discretizations and 
%determine if they are the same.
%
%As already noted, in order to describe the discretizations we are using 
%one only needs the following information:
%\begin{enumerate}
%\item Representing the Hilbert space $H$. Since $H$ is of the form
%  \[ H:=\cF_{\Field}(X,Z)\subset \cL^1_{\loc,\Field}(X,Z) \] 
%  it is a set of functions from the topological space $X$ into the 
%  vector space $Z$ (vector space over the field $\Field$), $H$ can be described by 
%  \begin{enumerate}
%  \item The dimension of $H$.
%  \item The underlying field $\Field$  as a set, so we do not bother to encode 
%    the field structure, we see $\Field$ only as a set.
%  \item The domain $X$ as a set, so we do not bother to encode the
%    topology, we see $X$ only as a set. 
%  \item The range $Z$ which is encoded as a set with an underlying 
%   field $\Field$, so we do not bother to encode 
%   the vector space structure of $Z$.
%  \end{enumerate}
%\item The dimension $n$ of the discretization.
%\item The family $V^n_{\rest}$ of sampling functions.
%\end{enumerate}

\section{Software implementation}

\subsection{General considerations}
%
\textbf{Discretizations:}
A discretization $\Discr_{n}(\Spc{X}) = (\Spc{X}, \Field^n, \RnX, \EnX)$ is implemented as discretized linear 
space $X_n$ in the following way:
%
\begin{itemize}
 \item $X_n$ is a (or derives from) \texttt{LinearSpace}.
 \item When initialized, it is provided with \emph{instances} of $\Spc{X}$ and $\Field^n$.
 \item It inherits the ``intersection'' of the structure of $\Spc{X}$ and $\Field^n$, i.e. if $\Spc{X}$ is a 
 \texttt{MetricSpace} and $\Field^n$ a \texttt{HilbertSpace}, $X_n$ will be a \texttt{MetricSpace}.
 \item Elements in $X_n$ are created by casting $n$-tuples of elements of $\Field$ (given as list, array, \ldots).
 \item $\RnX \colon \Spc{X} \to \Field^n$ 
   can optionally be provided as a \texttt{LinearOperator}. If $\Spc{X}$ is a \texttt{FunctionSpace},
   then an element $f_n \in X_n$ can be initialized from an analytically defined function $f \in \Spc{X}$ 
   (i.e., a Python function) as $f_n := \RnX(f)$, such as in the case of point collocation.
 \item $\EnX \colon \Field^n \to \Spc{X}$ can optionally be provided as a \texttt{LinearOperator}. 
   If $\EnX$ is given and $\Spc{X}$ is a \texttt{FunctionSpace}, 
   a discretized function $f_n \in X_n$ can be evaluated at any given point, e.g., by interpolation. 
   Such an interpolation rule does in this way define a continuous function $f \in \Spc{X}$.
\end{itemize}
%
\textbf{Re-discretizations:}
Consider two different discretizations of a fixed vector space $\Spc{X}$ given by 
\[ \Discr_{n}(\Spc{X}) = (\Spc{X}, \Field^n, \RnX, \EnX)
   \quad\text{and}\quad
  \Discr_{m}(\Spc{X}) =(\Spc{X}, \Field^m, \RmX, \EmX)
\]  
with $X_n$ and $X_m$ as the corresponding discretized linear spaces.
The re-discreti\-zation of $\Discr_{n}(\Spc{X})$ into $\Discr_{m}(\Spc{X})$ is the mapping 
\[  \DiscOp{Id}_{m,n} \colon \Field^m \to \Field^n
    \quad\text{defined as}\quad
    \DiscOp{Id}_{m,n} := \RmX \circ \EnX.
\]   
It is represented in the software as follows:
\begin{itemize}
 \item $\DiscOp{Id}_{m,n}$ is a \texttt{LinearOperator}\footnote{for now, may be relaxed to \texttt{Operator}} with 
 $\mathtt{domain} = X_n$ and $\mathtt{range} = X_m$.
 \item If $\EnX$ and $\RmX$ are present in $X_n$ or $X_m$, respectively, $\DiscOp{Id}_{m,n}$ is implemented as the 
 composition of these operators by default.
 \item A direct (optimized) implementation and can be provided, e.g., by an external software package.
\end{itemize}

\subsection{Dictionary-based discretization}
%

\subsection{Examples}

\begin{example}[Re-interpretation]
 Let $\bar f \in X_n = (\Spc{X}, \Field^n, \emptyset, \emptyset)$ be represented by $\valpha \in \Field^n$ and 
 $X_m = (\Spc{X}, \Field^m, \RmX, \EmX)$ with $m=n$. Then $\bar f$ can be re-interpreted as $\bar f \in \Spc{X}_m$ by 
 the identity mapping $\Op{I}: X_n \to \Spc{X}_m$. Practically, the re-mapped element will be aware of the ``added 
 structure '' $\RmX, \EmX$.
\end{example}





\cleardoublepage
\appendix
\section{Basic definitions}
\subsection{Algebraic structures}
We begin with the basic definition of the underlying algebraic objects, namely ring and field.
A field is basically an abstraction of the set $\Real$ of real numbers with the ``usual'' 
addition and multiplication.
\begin{definition}[Ring and field]
  A set $\Field$ is a \emph{ring} if there are two laws of composition
  $(\alpha,\beta)\mapsto \alpha+\beta$ and $(\alpha,\beta) \mapsto \alpha\cdot \beta$, called 
  respectively \emph{addition} and \emph{multiplication}, satisfying the following axioms:
  \begin{enumerate}
  \item $\Field$ is a commutative group under addition with zero element $0_{\Field}$, \ie,
    the following holds for all $\alpha,\beta,\gamma\in \Field$:
    \begin{enumerate}
    \item addition is associative, \ie, $\alpha + (\beta+\gamma)=(\alpha+\beta)+\gamma$,
    \item addition is commutative, \ie, $\alpha + \beta = \beta+\alpha$,
    \item the zero element is the identity element \wrt the addition so 
       $\alpha+0_{\Field}=0_{\Field}+\alpha=\alpha$,
    \item $-\alpha$ denotes the inverse of $\alpha$  \wrt the addition so 
       $\alpha+(-\alpha)=(-\alpha)+\alpha=0_{\Field}$.
    \end{enumerate}
  \item the multiplication is associative and possesses an identity element $1_{\Field}$, \ie,
    the following holds for all $\alpha,\beta,\gamma\in \Field$:
    \begin{enumerate}
    \item multiplication is associative, \ie, 
      $\alpha\cdot (\beta\cdot \gamma)=(\alpha\cdot \beta)\cdot \gamma$,
   \item $1_{\Field}$ is the identity element \wrt the multiplication so 
      $1_{\Field}\cdot \alpha=\alpha \cdot 1_{\Field}=\alpha$, 
  \end{enumerate}
  \item the multiplication is distributive with respect to the addition, \ie,
    for all $\alpha,\beta,\gamma\in \Field$,
    \[ (\alpha+\beta)\cdot \gamma = \alpha \cdot \gamma+\beta\cdot\gamma 
       \quad\text{and}\quad
       \alpha \cdot (\beta + \gamma) = \alpha \cdot \beta+\alpha\cdot\gamma.
    \]
  \end{enumerate}
  A ring $\Field$ is called a \emph{field} if it does not consist only of $0_{\Field}$ and every non-zero 
  element of $\Field$ has an inverse \wrt the multiplication. Finally, a field is \emph{commutative}
  if its multiplication is commutative.
\end{definition}
When working with fields we will simplify the notational burden by not explicitly writing 
the multiplication $\cdot$, \ie, $\alpha \beta$ means $\alpha\cdot\beta$ whenever 
$\alpha,\beta\in\Field$. Moreover, we will write $1$ and $0$ instead of $1_{\Field}$ and $0_{\Field}$.
Having defined the concept of a field, we are now ready to define the concept of a vector 
space.
\begin{definition}[Vector space]
  Let $\Field$ be a fixed field. Then a set $X$ is a \emph{left vector space over $\Field$} if 
  $X$ is a commutative group (the group law will be written additively in what follows)
  together with a map  $(\alpha,x) \mapsto \alpha\cdot x$, called the \emph{vector space 
  multiplication}, where
  \begin{enumerate}
  \item $\alpha \cdot (x+y)=\alpha \cdot x+\alpha \cdot y$
     for all $\alpha\in \Field$ and $x,y\in X$,
  \item $(\alpha + \beta) \cdot x= \alpha \cdot x + \beta\cdot x$ 
     for all $\alpha, \beta\in \Field$ and $x\in X$,\label{MII}
  \item $\alpha \cdot (\beta \cdot x) = (\alpha\beta)\cdot x$  
     for all $\alpha, \beta\in \Field$ and $x\in X$,\label{MIII}
  \item $1\cdot x = x$ for all $x\in X$.
  \end{enumerate}
  If \eqref{MIII} above is replaced by the axiom
  \[ \alpha \cdot (\beta \cdot x) = (\beta \alpha)\cdot x
        \quad\text{for all $\alpha, \beta\in \Field$ and $x\in X$,}
  \]
  then we say that $X$ is a \emph{right vector space over $\Field$}. Elements in the 
  field $\Field$ are called \emph{scalars} and elements in $X$ are called \emph{vectors}. 
  Finally,  note that in \eqref{MII} above, the addition $+$ in the left hand side 
  represents the addition in 
  the field $\Field$, whereas in the right hand side it represents the group law in $X$.
\end{definition}  
We say that a subset $V \subset X$ is a \emph{vector subspace of $X$} if $V$ itself is a vector 
space over $\Field$ with the same operations as in $X$. This is equivalent to requiring that 
$0\in V$ and $\alpha x, x+y \in V$ whenever $x,y\in V$ and $\alpha \in \Field$.

Often the vector space multiplication $\cdot$ is not written explicitly. In fact, if $X$ is a 
left (resp.\@ right) vector space over a field $\Field$, then any expression of the type 
$\alpha x$ where $\alpha\in\Field$ and $x\in X$ means $\alpha \cdot x$. Moreover, we will not 
notationally distinguish between the addition in $\Field$ and the group law in $X$, so 
$\alpha + \beta$ refers to the addition in $\Field$ if $\alpha,\beta\in \Field$ and to the group law in 
$X$ if $\alpha,\beta\in X$. In the same way we do not notationally distinguish between the 
zero in $X$ and the zero in $\Field$

Having dealt with the notational conventions, let us now proceed by defining concepts such 
as linear combination, basis, and dimension.
\begin{definition}[Linear combination and independence]
   Let $X$ be a right (resp.\@ left) vector space over a field $\Field$. Also, 
   let $\{ v_1, \ldots, v_{n} \}  \subset X$ be a fixed subset. An element $x\in X$ 
   is said to be a \emph{linear combination of the elements $\{ v_1, \ldots, v_{n} \}$} if 
   there exists a subset $\{ \alpha_1,\ldots,\alpha_{n} \} \subset \Field$ such that 
   \[  x= \alpha_1 v_1 + \ldots +  \alpha_{n} v_{n}. \]
   The elements $\{ \alpha_1,\ldots,\alpha_{n} \}$ are then called the \emph{coefficients 
  (or coordinates)  of $x$ \wrt $\{ v_1, \ldots, v_{n} \}$}. Moreover, we say that this finite 
  set is \emph{linearly independent} if 
   \[ \alpha_1 v_1 + \ldots + \alpha_{n} v_{n} =0
      \quad\text{implies that}\quad \alpha_1=\ldots=\alpha_{n}=0. \]
  Note that the zero in the left hand side is the zero vector in $X$ whereas the zero in the right 
  hand side is the zero in the field $\Field$.
\end{definition}
Using the axiom of choice, one can show, see, \eg, Chapter~II, section~7 in \cite{Bo89}, 
that every vector space $X$ has a maximal linearly independent subset $B \subset X$ 
spanning $X$, \ie, every vector $x\in X$ can be written as a finite linear combination 
of the elements in $B$. Such a maximal subset is called a \emph{(Hamel) basis}. If 
$B$ denotes a basis of the vector space $X$, then any element $x\in X$ has a 
unique representation as a linear combination of vectors $v_1,\ldots,v_{n}\in B$, \ie, there 
exists unique scalars $\alpha_1,\ldots,\alpha_{n}\in\Field$ such that 
\[ x = \alpha_1 v_1 + \ldots + \alpha_{n} v_{n}. \]
The scalars $\alpha_1,\ldots,\alpha_{n}$ are then called the  \emph{coordinates/coefficients}
of $x$ with respect to the basis $B$. The \emph{dimension} of $X$ is defined as the cardinality 
of its Hamel basis. Since one can show that any two Hamel basis of $X$ 
have the same cardinality, the concept of dimension is well-defined.
\begin{definition}[Linear map]
  Let $X$ and $Y$ be two (left) vector spaces over the same field $\Field$. A map
  $f \colon X \to Y$ is called a \emph{linear map (homomorphism)} if
  \[  f(x+y)=f(x)+f(y) \quad\text{and}\quad
      f(\alpha x)=\alpha f(x)  \]
  for all $x,y\in X$ and $\alpha\in \Field$.
  Note that $\alpha x$ refers to the vector space multiplication in $X$, whereas $\alpha f(x)$
  refers to the vector space multiplication in $Y$.
  If $X$ and $Y$ are two (right) vector spaces over the same field $\Field$, then we need to
  replace the condition $f(\alpha x)=\alpha f(x)$ with $f(x\alpha)= f(x) \alpha$.
  Finally,  we let  $\Hom_{\Field}(X,Y)$ denote the set of linear mappings from $X$ into $Y$.
\end{definition}   
Since any result that holds for left vector spaces over $\Field$ also holds for right vector spaces 
over $\Field$, we shall in the sequel we shall drop the prefix left (resp.\@ right). Moreover, if
the field is commutative, then there is no difference in the sense that any left vector spaces
is also a right vector spaces.

An important class of vector spaces are the product spaces. 
If $X_1, \ldots, \Field^{n}$ are  vector spaces over the same field $\Field$, then we
define the \emph{product space}
\begin{equation}\label{eq:ProdSpace}
   X:= X_1 \times \ldots  \times \Field^{n} 
\end{equation}
as the set of elements $(x_1,\ldots,x_{n})$ where $x_i\in X_i$. There is a natural vector 
space structure on $X$, namely, if $(x_1,\ldots,x_{n}), (y_1,\ldots,y_{n})\in X$ and 
$\alpha\in\Field$, then 
\begin{align*}
   (x_1,\ldots,x_{n})+(y_1,\ldots,y_{n}) &:= (x_1+y_1,\ldots,x_{n}+y_{n}) \\
   \alpha (x_1,\ldots,x_{n}) &:= (\alpha x_1,\ldots,\alpha x_{n}).
\end{align*}
Note that $x_i+y_i$ and $\alpha x_i$ refer to the group law and vector space 
multiplication in the vector space $X_i$. It is easy to show that with the above definitions, 
$X$ becomes a vector space over $\Field$. 
For notational simplicity, in many cases one would like to simplify the notation for
elements in a product space $X$. 
%In fact, we will denote such elements by the corresponding 
%bold face letter, \ie, $(x_1,\ldots,x_{n})$ is denoted 
%by $\bx$ and $(y_1,\ldots,y_{n})$ by $\by$, \etc Thus, with this notational convention 
%$\bx_i$ refers to an element in the product space $X$ that is indexed by $i$ and 
%$x_i$ refers to the corresponding element in $X_i$ of $\bx=(x_1,\ldots,x_{n})$. 

Now, if $X$ is a product space defined as in \eqref{eq:ProdSpace}, then the 
\emph{natural projections}
\[ \pi_i \colon X \to X_i \quad\text{are defined as}\quad \pi_i(x):=x_i, \]
and they are surjective linear mappings, \ie, $\pi_i\in\Hom_{\Field}(X,X_i)$. 
Now, it is easy to show, see, \eg, Proposition~4, Section~1.5, Chapter~II in \cite{Bo89}, 
that for any vector space $Y$ over $\Field$ and any finite set of linear mappings 
$f_i \in \Hom_{\Field}(Y,X_i)$,  $i=1,\ldots,n$, there exists a unique map 
$f\in \Hom_{\Field}(Y,X)$ such that 
\[ \pi_i \circ f = f_i. \]
Conversely, given $f\in \Hom_{\Field}(Y,X)$, we can always define $f_i \in \Hom_{\Field}(Y,X_i)$ by 
the above relation. 
%We will use the bold face notation $\vf$ on the elements in $\Hom_{\Field}(Y,X)$
%\emph{only} when there is an explicit need to refer to the $f_i$:s.

We will finally define the concepts of norm and inner product for vector spaces over $\Real$ 
or $\Complex$. 
\begin{definition}[Norm and inner product]
  Let $\Field$ be a field of real numbers $\Real$ or complex numbers $\Complex$  with the usual 
  addition and multiplication. Moreover, let $X$ be a vector space over $\Field$. A mapping 
  \[ \Vert \cdot \Vert \colon X \to [0,\infty[ \]
  is called a \emph{norm on $X$} if the following holds for all $x,y \in X$ and $\alpha\in\Field$:
  \begin{enumerate}
  \item $\Vert \alpha x \Vert = \vert \alpha \vert \, \Vert x \Vert$,
  \item $\Vert x \Vert \geq 0$ with equality if and only if $x$ is the zero element in $X$,
  \item $\Vert x+y \Vert \leq \Vert x \Vert+\Vert y \Vert$.
  \end{enumerate}
  A mapping 
  \[ \ip{\cdot}{\cdot} \colon X\times X \to \Field \]
  is called the \emph{inner product on $X$}
  if the following holds for all $x,y,z \in X$ and $\alpha\in\Field$:
  \begin{enumerate}
  \item $\ip{x}{y}=\overline{\ip{y}{x}}$,
  \item $\ip{\alpha x}{y}=\overline{\alpha}\ip{x}{y}$ and $\ip{x}{\alpha  y}= \alpha \ip{x}{y}$,
  \item $\ip{x + y}{z} = \ip{x}{z}  +  \ip{y}{z}$,
  \item  $\ip{x}{x}\geq 0$ with equality if and only if $x$ is the zero element in $X$.
  \end{enumerate}
  A vector space with an inner product is called an \emph{inner product (or pre Hilbert) space}
  and a vector space with a norm is called a \emph{normed space}.
\end{definition}
Let $X$ be an inner product space. Then we say that $x,y \in X$ are \emph{orthogonal} 
if $\ip{x}{y}=0$. 


\paragraph{Examples of vector spaces.}
The first example is a trivial one, namely that any field $\Field$ is a one-dimensional 
vector space over itself. The next example is the product space $\Field^{n}$ which is  defined as 
the $n$ fold product of $\Field$, \ie,
\[ \Field^{n} := \underbrace{\Field \times \ldots \times \Field}_{\text{$n$ times}}. \]
Then $\Field^{n}$ is an $n$-dimensional vector space 
over $\Field$. In this setting, $\Field^1$ is isomorphic to $\Field$, so we will identify $\Field^1$ with $\Field$
without further notice. The importance of the vector space $\Field^{n}$ originates from the fact 
that it is the canonical $n$-dimensional vector space. In fact, any $n$-dimensional vector 
space $X$ can be naturally identified 
with $\Field^{n}$. Let  $\{ v_1,\ldots, v_{n} \}$ be a basis of $X$ and define the map 
$\phi \colon X \to \Field^{n}$ as
\[  \phi\Bigl( \sum_{i=1}^n \alpha_i v_i \Bigr) := (\alpha_1,\ldots,\alpha_{n})\in\Field^{n}
    \quad\text{so}\quad 
    \phi^{-1}(\alpha_1,\ldots,\alpha_{n}) = \sum_{i=1}^n \alpha_i v_i.
\]
The above map $\phi$ is an vector space isomorphism which we call the \emph{natural 
isomorphism of $X$ \wrt the basis $\{ v_1,\ldots, v_{n} \}$}, so $X$ is isomorphic to $\Field^{n}$.
Let us also mention that in order to comply with the notation used in 
matrix algebra\footnote{Matrix, and more generally, tensor algebra for matrices and tensors with 
elements in a general field $\Field$ is a straightforward generalization of the matrix and 
tensor algebra for matrices and tensors with elements in $\Real$. 
We refer to \cite[Chapter~II, Section~10 and Chapter~III]{Bo89} for the formal details.} 
in $\Field$, a vector $\alpha\in\Field^{n}$ is written as a column vector of its components, 
\ie,
\[ \alpha = \begin{pmatrix} \alpha_1 \\  \vdots \\  \alpha_{n} \end{pmatrix}
   \quad\text{with $\alpha_j \in \Field $ for $j=1,\ldots,n$.}
\]

As a final example,  let $X$ be a set and $Y$ a vector space over a field $\Field$. Now, the 
set $\Map(X,Y)$ of all mappings from $X$ into $Y$ has a natural vector space 
structure. In fact, for $f,g\in \Map(X,Y)$ and $\alpha\in\Field$ we define $f+g$ and $\alpha f$ as 
\[ (f+g)(x):= f(x)+g(x) \quad\text{and}\quad (\alpha f)(x):=\alpha f(x) \]
for all $x\in X$. Then, whenever $f+g, \alpha f\in  \Map(X,Y)$, the set $\Map(X,Y)$ becomes
a vector space over $\Field$ under the above definitions of the group law and vector space 
multiplication. An interesting special case is when $X$ is a vector space over $\Field$ and 
$\Map(X,Y)= \Hom_{\Field}(X,Y)$.


%\subsection{Topological structures}
%Let $\Field$ denote the field of real numbers $\Real$ or complex numbers $\Complex$ with the usual
%addition and multiplication. 

\bibliographystyle{plain}
\bibliography{discretization}

\end{document}
