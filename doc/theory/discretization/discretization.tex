\documentclass[a4paper]{article}
\usepackage{amsmath,amsfonts}
\usepackage[matrix,arrow]{xy}
\usepackage{extraenv}

\newcommand{\R}{\mathbb{R}}
\newcommand{\C}{\mathbb{C}}
\newcommand{\N}{\mathbb{N}}
\newcommand{\cX}{\mathcal{F}}
\newcommand{\Cc}{\mathcal{C}}
\newcommand{\cH}{\mathcal{H}}
\newcommand{\loc}{\text{loc}}
\newcommand{\bxi}{\boldsymbol{\xi}}
\newcommand{\bx}{\boldsymbol{x}}
\newcommand{\bb}{\boldsymbol{b}}
\newcommand{\by}{\boldsymbol{y}}
\newcommand{\bz}{\boldsymbol{z}}
\newcommand{\bA}{\boldsymbol{A}}
\newcommand{\cF}{\mathcal{F}}
\newcommand{\cL}{\mathcal{L}}
\newcommand{\K}{\mathbb{K}}
\newcommand{\veps}{\varepsilon}
\DeclareMathOperator{\prob}{Prob}
\DeclareMathOperator{\linspan}{linspan}
\newcommand{\ent}{\text{ent}}
\newcommand{\ext}{\text{ext}}
\newcommand{\true}{\text{true}}
\newcommand{\rest}{\text{rest}}
\newcommand{\norm}{\text{norm}}
\newcommand{\ip}[2]{\langle #1,#2 \rangle}  
\newcommand{\wt}[1]{\widetilde{#1}}
\newcommand{\optim}[2]
  {  \begin{split}
       & #1  \\
       & #2
     \end{split}
  }

\title{The object classes and the discretization problem}
\author{Ozan \"Oktem}

\begin{document}
\maketitle

I begin with the general mathematical description of 
the problem, then I give some examples, and finally I pose the software 
related problems that we face.

\section{The general dicretization problem}
We will describe the general theory of discretization following 
\cite[Chapter~34]{ZeIIB85}. Often one deals with mappings between 
two infinite dimensional spaces. In order to numerically work with such 
mappings, one must reduce the infinite dimensional spaces to finite 
dimensional counterparts. This reduction of dimensionality is 
referred to as a \emph{discretization}. 

More formally, let $H$ be a Hilbert space over a field $\K$. 
An \emph{$n$-dimensional 
discretization of $H$} is determined by the tuple
\[ \bigl\{ H,H_{n},R_{n},E_{n} \bigr\} \]
where $H_{n}$ is an $n$-dimensional Hilbert space (usually over $\K$) and 
\[ R_{n} \colon H \to H_{n}
   \quad\text{and}\quad
   E_{n} \colon H_{n} \to H. \]
The maps $R_{n}$ and $E_{n}$ are called the  \emph{restriction} and 
\emph{extension} maps associated to the discretization. 
\begin{Def}
  The sequence of $n$-dimensional discretizations $\bigl\{ 
  H,H_{n},R_{n},E_{n} \bigr\}_{n}$ of $H$ is \emph{consistent} whenever
  \[ \lim_{n\to \infty} 
     \bigl\Vert ( E_{n}\circ R_{n})(f)-f \bigr\Vert_{H}\to 0
     \quad\text{for all $f\in H$.}\]
  Let $H$ and $H'$ be Hilbert spaces over the fields $\K$ and $\K'$ 
  and $T \colon H \to H'$. Also, let 
  \[ \bigl\{ H,H_{n},R_{n},E_{n} \bigr\} 
     \quad\text{and}\quad
     \bigl\{ H',H_{m},R_{m},E_{m} \bigr\}\]
  denote $n$- and $m$-dimensional discretizations of $H$ and $H'$, 
  respectively. Then a map $T_{n,m} \colon H_{n} \to H'_{m}$ is 
  \emph{consistent with the above discretizations of $H$ and $H'$} 
  whenever the following diagram commutes:
  \[ \xymatrix{
      H \ar[rr]_{T} \ar@<0.3ex>[dd]_{E_{n}} & &  
      H' \ar@<0.3ex>[dd]_{E_{m}} \\
       & &  \\
      H_{n} \ar@<0.3ex>[uu]_{R_{n}} \ar[rr]_{T_{n,m}}  & &     
      H'_{m} \ar@<0.3ex>[uu]_{R_{m}}  }
  \]
  In that case we say that $T_{n,m}$ is the \emph{discretization of $T$}.
\end{Def}  
Thus, if we have a sequence $\bigl\{ H,H_{n},R_{n},E_{n} \bigr\}_{n}$ 
of consistent $n$-dimensional discretizations of $H$, 
then for large $n$ one can make the approximation that $E_{n}\circ 
R_{n}$ is the identity operator i.e.\@
\begin{equation}\label{eq:fApprox}
    f \approx E_{n}(f_{n}) \quad\text{where}\quad
    f_{n}:=R_{n}(f).
\end{equation}  
Also, given $T$ one can construct a natural discretization $T_{n,m}$ 
by defining $T_{n,m}:= E_{n} \circ T \circ R_{m}$.

We consider a simple example which is the Galerkin method in Hilbert 
space. Let $\{ \phi_{i} \}_{i}\subset H$ be a 
complete orthonormal system in the separable Hilbert space $H$. We 
then set 
\[ H_{n} := \linspan \{\phi_{1},\ldots,\phi_{n} \}
   \quad\text{for $n\in\N$.}
\]
Let $R_{n} \colon H \to H_{n}$ be the orthogonal projection operator 
from $H$ onto $H_{n}$, i.e.\@
\[ R_{n}(f)=\sum_{i=1}^n \ip{\phi_{i}}{f}_{H}\phi_{i}
   \quad\text{for all $f\in H$.} \]
Moreover, let $E_{n}\colon H_{n} \to H'_{n}$ be the embedding operator 
corresponding to $H_{n}\subset H$. Now consider an
operator $T \colon H \to H$ (so $H':=H$ and $H'_{n}:=H_{n}$). Then 
$R_{n}(f)\to f$ as $n\to \infty$, so the discretizations of $H$ are 
compatible. Moreover, the discretization $T_{n,n}$ of $T$ where 
\[ T_{n,n}:=R_{n}\circ T \circ E_{n} \]
is consistent with the discretization of $H$ and the equation
$T(f)=g$ can be replaced with 
\[ \ip{T(f_{n})}{\phi_{j}}_{H}=\ip{g}{\phi_{j}}_{H}
   \quad\text{$f_{n}\in H_{n}$ and $j=1,\ldots,n$.} \]
Also, note that it is not necessary for $H_{n}$ and $H'_{m}$ to be 
subsets of $H$ and $H'$. In fact in most of the situations this is not the 
case. 

\paragraph{Discretization of functions spaces.}
We begin by introducing some useful notation that well be used 
throughout this section.
\begin{Not}
  Let $X$ be a topological space, $\mu$ a fixed 
  Borel measure, $\K$ a field, and $Z$ a vector space over $\K$. 
  Then we define $\cL^1_{\loc,\K}(X,Z)$ as the Hilbert space over $\K$ of
  locally $\mu$-measurable functions 
  $f\colon X\to Z$. The notation $\cL^1(X)$ simply stands for 
  $\cL^1_{\loc,\R}(X,\R)$.
\end{Not}  
Throughout this section $Z$ and $W$ are vector spaces over the 
fields $\K$ and $\K'$ and $X, Y$ are topological spaces.
Also, let
\[ H:=\cF_{\K}(X,Z)\subset \cL^1_{\loc,\K}(X,Z) \]
denote a  Hilbert subspace. Below we consider a specific family of 
discretizations of $H$  containing all the cases we are 
interested in. 
\begin{Def}
  Let $\cH:=\{ \phi_{1},\ldots,\phi_{m} \}\subset H$ be a 
  fixed finite set of $m$ functions.
  Then we make the following definitions:
  \begin{enumerate}
  \item The \emph{restriction operator $R(\cdot;\cH)$ induced by 
    $\cH$} is the  mapping  
    \[ R_{m}(\cdot;\cH) \colon H \to Z^m \]
    where 
    \[ R_{m}(f;\cH):=\bigl( r_{1}(f;\cH),\ldots,r_{m}(f;\cH) \bigr) 
       \quad\text{for all $f\in H$,} \]
    with $r_{i}(f;\cH)\in Z$ for $i=1,\ldots,m$.
    We do not further specify how $r_{i}(\cdot;\cH)$ is defined.
    Simply note that two choices of $r_{i}(\cdot;\cH)$ are common, namely
    $r_{i}(f;\cH):=f(x_{i})$ where $x_{i}\in X$ is a fixed point (a sample 
    point) and $r_{i}(f;\cH):=\ip{f}{\phi_{i}}_{H}$ which is natural whenever
    $\cH$ is a subset of an orthonormal basis in $H$. Also, 
    $r_{i}(\cdot;\cH)$ is linear in both these cases.
    In this context the elements in $\cH$ are called the 
    \emph{sampling functions}.
  \item The \emph{extension operator $R(\cdot;\cH)$ induced by 
    $\cH$} is defined as the mapping  
    \[ E_{m}(\cdot;\cH) \colon Z^m \to H \]
    where 
    \[ E_{m}(\bb;\cH):= x\mapsto \sum_{i=1}^m b_{i}\phi_{i}(x)
       \quad\text{for all $\bb\in Z^m$.} \]
    In this context the elements in $\cH$ are called the 
    \emph{extrapolation functions}.       
  \end{enumerate}
\end{Def}
Note that the sets of sampling and extrapolation functions do not 
need to coincide.  The only requirement is that the number of 
elements $m$ must be the same. Thus, a pair 
$(\cH^m_{\rest},\cH^m_{\ext})$ of $m$ 
sampling and extrapolation functions defines an $m$-dimensional 
discretization of $H$ by the tuple
\[ \bigl\{
     H,Z^m,R_{m}(\cdot;\cH^m_{\rest}),E_{m}(\cdot;\cH^m_{\ext})
   \bigr\}. \]
   
Now, consider two the following two pairs $(\cH^n_{\rest},\cH^n_{\ext})$ and 
$(\cH^m_{\rest},\cH^m_{\ext})$ of $n$ and $m$ sampling and extrapolation 
functions:
\begin{alignat*}{2}
  \cH^n_{\rest} &:=\{ \phi_{1},\ldots,\phi_{n} \} &\qquad
  \cH^m_{\rest} &:=\{ \psi_{1},\ldots,\phi_{m} \}  \\
  \cH^n_{\ext} &:=\{ \wt{\phi}_{1},\ldots,\wt{\phi}_{n} \} &\qquad
  \cH^m_{\ext} &:=\{ \wt{\psi}_{1},\ldots,\wt{\psi}_{m} \}    
\end{alignat*}
Each pair yields a discretization of $\cF_{\K}(X,Z)$ by the tuples
\[ 
   \bigl\{
     \cF_{\K}(X,Z),Z^n,R_{n}(\cdot;\cH^n_{\rest}),E_{n}(\cdot;\cH^n_{\ext})
   \bigr\} \]
and
\[ \bigl\{
     \cF_{\K}(X,Z),Z^m,R_{m}(\cdot;\cH^m_{\rest}),E_{m}(\cdot;\cH^m_{\ext})
   \bigr\}. \] 
What is the relation between these two discretizations? A function 
$f\in \cF_{\K}(X,Z)$ is represented by the ``vectors'' 
\[ \bz_{n}:= R_{n}(f;\cH^n_{\rest})\in Z^n \quad\text{and}\quad
   \bz_{m}:= R_{m}(\cdot;\cH^m_{\rest})\in Z^m \]
and we seek the relaation between $\bz_{n}$ and $\bz_{m}$. 
Now making use of the approximation $f\approx 
E_{n}(\bz_{n};\cH^n_{\ext})$, i.e.\@
\[ f \approx x \mapsto 
       \sum_{i=1}^n z_{n,i} \wt{\phi}_{i}(x)
   \quad\text{in $\cF_{\K}(X,Z)$,} 
\]
which implies that
\begin{align*}
  \bz_{m} &:=R_{m}(f;\cH^m_{\rest}) =
    \bigl( r_{1}(f,\cH^m_{\rest}),\ldots,r_{m}(f,\cH^m_{\rest}) \bigr) \\
    &\approx 
      \Bigl( r_{1}\bigl(E_{n}(\bz_{n},\cH^n_{\ext}),\cH^m_{\rest}\bigr),
             \ldots,
	     r_{m}\bigl(E_{n}(\bz_{n},\cH^n_{\ext}),\cH^m_{\rest}\bigr)
      \Bigr).
\end{align*}
When $r_{k}(\cdot,\cH^m_{\rest})$ are linear, then 
\[ r_{k}\bigl(E_{n}(\bz_{n},\cH^n_{\ext}),\cH^m_{\rest}\bigr)=
     r_{k}\biggl(\sum_{i=1}^n 
       z_{n,i}\wt{\phi}_{i}(\cdot),\cH^m_{\rest}\biggr)=
     \sum_{i=1}^n z_{n,i} r_{k}(\wt{\phi}_{i},\cH^m_{\rest}).
\]   
Thus, for $k=1,\ldots,m$
\[
  z_{m,k}\approx 
    \sum_{i=1}^n z_{n,i} r_{k}(\wt{\phi}_{i},\cH^m_{\rest}).
\]    
and defining 
\[ A_{i,k}:= r_{k}(\wt{\phi}_{i},\cH^m_{\rest})
   \text{ for $i=1,\ldots,n$ and $j=1,\ldots,m$,} \]
enables us to write
\[ \bz_{m}=\bz_{n}\cdot\bA
   \quad\text{where } 
   \bA:=[ A_{i,k} ]_{\substack{1\leq i \leq n \\ 1\leq k \leq m}}
   \text{ is an $(n\times m)$-matrix.} \]
Conversly, defining 
\[ \wt{A}_{k,i}:=r_{i}(\phi_{k},\cH^n_{\rest})
   \text{ for $i=1,\ldots,n$ and $k=1,\ldots,m$,} \]
enables us to write
\[ \bz_{n}=\bz_{m}\cdot\wt{\bA}
   \quad\text{where } 
   \wt{\bA}:=[ \wt{A}_{i,k} ]_{\substack{1\leq k \leq M \\ 
   1\leq i \leq n}}
   \text{ is an $(m\times n)$-matrix.} \]
Finally, if $\cH^n_{\rest}=\cH^n_{\rest}$ and 
$\cH^m_{\ext}=\cH^m_{\rest}$, then $\bA=\wt{\bA}^T$.   


\section{Relation to inverse problems}
Let $X, Y$ be topological spaces and $T \colon \cF(X,\K) \to 
\cF(Y,\K')$. The inverse problem we are interested in is to find 
$f\in H_{X}:=\cF(X,\K)$ that solves the operator equation $T(f)=g$ where 
$g\in H_{Y}:=\cF(Y,\K')$ is 
known only on a subset $Y_{0}\subset Y$. In reality $Y_{0}$ is a 
finite set, so $g$ is known only at a finite number of points in $Y$. 

This enables us to replace the continuous inverse  is given, with a sequence of 
discrete inverse problems of finding $f_{n}\in H_{n}$ such that 
$T_{n,m}(f_{n})=R_{m}(g)$.

Finally, note that the dimension of the discretization of  $\cF(Y,W)$ 
occurring in the inverse problem in subsection~\ref{sec:Inv} is naturally 
determined by the fact that $T(f_{\true})$ is known (up to measurement 
errors) on a given set $Y_{0}\subset Y$ with $m$ elements. 
It is natural to choose an $m$-dimensional discretization of 
$\cF(Y,W)$ where the choice of the sampling and extrapolation functions are 
related to the elements in $Y_{0}$.



   
\section{The object model}
What we are looking for is an efficient object structure representing 
the discretization problem described above. The idea is that given an 
operator 
\[ T \colon \cF(X) \to \cF(Y) \]
the object structure in the code must represent the discretization to 
the level of accuracy that enables one to change between different
discretizations $T_{N,M}$ and $T_{N',M'}$ of $T$, at least 
for different sample based discretizations.

The naive way to model the object structure is to fully replicate the 
mathematical setting (fields, vector spaces over fields, normed spaces, Hilbert 
spaces over fields, etc) but this produces a very large object structure 
that we believe is not feasible nor necessary. The idea that we have 
played with
is to have an object structure that is general enough to capture different 
choices of sample based discretizations and relations in between them 
without being inefficient. 
As an example, what possible function spaces $\cF(X)$ do we have? Well, 
mathematically this depends on $T$ and examples for $X=\R^n$ 
are $\Cc_{c}(\R^n)$ (continuous functions with compact 
support), $\Cc^\infty_{c}(\R^n)$ (smooth functions with compact 
support), and 
$\mathcal{S}(\R^n)$ (functions in the Schwartz class, i.e.\@ fast 
decaying smooth functions). When using 
a sample based discretization it is actually not relevant exactly which 
Hilbert space that one started with. The only important factor is the 
underlying field and the inner product (which affects the inner 
product in the finite dimensional representation) since the same 
sample based discretization yields the same finite dimensional representation for 
all Hilbert spaces over the same field with the same inner product structure.
As an example a fixed sample based discretization can not differ between bounded 
integrable functions or 
continuous functions if the inner product that we use is the same. 
Hence, a Hilbert space may be represented by the underlying field 
($\R$, $\C$,\ldots), and the inner product structure (or at least its 
discrete counterpart).

\subsection{Consistency checks}
It is desirable that the object model can check for consisteny. The 
first problem is to be able to compare two discretizations and 
determine if they are the same.

As already noted, in order to describe the discretizations we are using 
one only needs the following information:
\begin{enumerate}
\item Representing the Hilbert space $H$. Since $H$ is of the form
  \[ H:=\cF_{\K}(X,Z)\subset \cL^1_{\loc,\K}(X,Z) \] 
  it is a set of functions from the topological space $X$ into the 
  vector space $Z$ (vector space over the field $\K$), $H$ can be described by 
  \begin{enumerate}
  \item The dimension of $H$.
  \item The underlying field $\K$  as a set, so we do not bother to encode 
    the field structure, we see $\K$ only as a set.
  \item The domain $X$ as a set, so we do not bother to encode the
    topology, we see $X$ only as a set. 
  \item The range $Z$ which is encoded as a set with an underlying 
   field $\K$, so we do not bother to encode 
   the vector space structure of $Z$.
  \end{enumerate}
\item The dimension $n$ of the discretization.
\item The family $\cH^n_{\rest}$ of sampling functions.
\end{enumerate}

\begin{thebibliography}{99}
\bibitem{ZeIIB85} Zeidler, E., Nonlinear Functional Analysis and its 
  Applications~II/B, Springer Verlag, 1985
\end{thebibliography}  

\end{document}
